%  This is an example LaTeX file. The percent sign is used to mark the
% start of a comment.
%
%  - Michael Weeks,  January, 2003
%
\documentclass[10pt]{article}
\usepackage[a4paper, total={6in, 8in}]{geometry}
\usepackage{textgreek}
\usepackage{siunitx}
\usepackage{rotating,graphicx}
\usepackage[hidelinks]{hyperref}
\usepackage{pdflscape}
\usepackage{float}
\usepackage{subfig}
\usepackage{listings}
\hypersetup{
  colorlinks   = true,    % Colours links instead of ugly boxes
  urlcolor     = blue,    % Colour for external hyperlinks
  linkcolor    = blue,    % Colour of internal links
  citecolor    = red      % Colour of citations
}
% \setlength{\parskip}{\baselineskip}%
% \setlength{\parindent}{1.5pt}%
\usepackage[acronym]{glossaries}
\makeglossaries

\newacronym{edf}{EDF}{european data format}
\newacronym{gui}{GUI}{graphical user interface}
\newacronym{stft}{STFT}{short time fourier transform}

% %  This is an example LaTeX file. The percent sign is used to mark the
% start of a comment.
%
%  - Michael Weeks,  January, 2003
%
\documentclass[10pt]{article}
\usepackage[a4paper, total={6in, 8in}]{geometry}
\usepackage{textgreek}
\usepackage{siunitx}
\usepackage{rotating,graphicx}
\usepackage[hidelinks]{hyperref}
\usepackage{pdflscape}
\usepackage{float}
\usepackage{subfig}
\hypersetup{
  colorlinks   = true,    % Colours links instead of ugly boxes
  urlcolor     = blue,    % Colour for external hyperlinks
  linkcolor    = blue,    % Colour of internal links
  citecolor    = red      % Colour of citations
}
% \setlength{\parskip}{\baselineskip}%
% \setlength{\parindent}{1.5pt}%
\usepackage[acronym]{glossaries}
\makeglossaries

\newacronym{edf}{EDF}{european data format}
\newacronym{gui}{GUI}{graphical user interface}
\newacronym{stft}{STFT}{short time fourier transform}

% %  This is an example LaTeX file. The percent sign is used to mark the
% start of a comment.
%
%  - Michael Weeks,  January, 2003
%
\documentclass[10pt]{article}
\usepackage[a4paper, total={6in, 8in}]{geometry}
\usepackage{textgreek}
\usepackage{siunitx}
\usepackage{rotating,graphicx}
\usepackage[hidelinks]{hyperref}
\usepackage{pdflscape}
\usepackage{float}
\usepackage{subfig}
\hypersetup{
  colorlinks   = true,    % Colours links instead of ugly boxes
  urlcolor     = blue,    % Colour for external hyperlinks
  linkcolor    = blue,    % Colour of internal links
  citecolor    = red      % Colour of citations
}
% \setlength{\parskip}{\baselineskip}%
% \setlength{\parindent}{1.5pt}%
\usepackage[acronym]{glossaries}
\makeglossaries

\newacronym{edf}{EDF}{european data format}
\newacronym{gui}{GUI}{graphical user interface}
\newacronym{stft}{STFT}{short time fourier transform}

% %  This is an example LaTeX file. The percent sign is used to mark the
% start of a comment.
%
%  - Michael Weeks,  January, 2003
%
\documentclass[10pt]{article}
\usepackage[a4paper, total={6in, 8in}]{geometry}
\usepackage{textgreek}
\usepackage{siunitx}
\usepackage{rotating,graphicx}
\usepackage[hidelinks]{hyperref}
\usepackage{pdflscape}
\usepackage{float}
\usepackage{subfig}
\hypersetup{
  colorlinks   = true,    % Colours links instead of ugly boxes
  urlcolor     = blue,    % Colour for external hyperlinks
  linkcolor    = blue,    % Colour of internal links
  citecolor    = red      % Colour of citations
}
% \setlength{\parskip}{\baselineskip}%
% \setlength{\parindent}{1.5pt}%
\usepackage[acronym]{glossaries}
\makeglossaries

\newacronym{edf}{EDF}{european data format}
\newacronym{gui}{GUI}{graphical user interface}
\newacronym{stft}{STFT}{short time fourier transform}

% \include{monthly_report.glsdefs}
% \include{monthlly_report.acn}

%\journal{CSc 4110 Final Report}

%\title[journalExample]{Format for Project Reports}
\title{
  An update on the project: 
  \textbf{
      \textit{
        Development of an Automatic Instrument for Schizophrenia(SZ) Diagnosis
        }
      }, for the MCIP Innovation Prize 2022.
  }
% \author{
% Emmanuel OLATEJU \\
%     \begin{affiliation}
%       Supervised by Dr. K.P. Ayodele \\ 14/02/2023, \\
%       email: \mbox{kayodele@gmail.com, eoolateju@student.oauife.edu.ng}
%     \end{affiliation}
% }

\begin{document}
\maketitle

\section{Summary}
The previous report presented the data acquisition and processing methods along some obtained 
results. Also highlighted were technical issues noticed during data acquisition and 
issues with ergonomics of developed software for the clinicians. The previous report also 
presented the next steps to be taken.\\
\\
This report will highlight the challenges and next steps stated within the previous eport, then 
then present the progress made from the time of submission of the last report till the moment of 
submission of this report, more recent challenges and the next steps to be taken.

\section{From February's Report}
The last report presented results on some extracted features and stated some challenges the project is 
facing which include scheduling and mobility issues, 
subject attitude towards participation in exercise, cue communication and dellivery method, absence of 
clinician, subject feedback system, fuzzy entropy spatial complexity algoritm problems, non-uniform session times.\\
\\
Also stated within the last report were the set of steps to be taken next. This steps included establishing 
the best pre-processing path in terms of features being more discriminable, resolving left over issues 
with audio cue delivery mechanism, development of hand held annotator for taking feedback from clinician and 
subject, developing montage analysis algorithm, improvement of data acquisition paradigm and more data acquisition.\\
\\
In the next section, what has been achieved is discussed.

\section{Progress}
In order to establish uniform data acquisition paradigm so that processing of data can be made 
easier, from the already acquired data, the minumum permissible timeframe that can be met from 
the sessions data of all subjects was adopted for data processing and incroporated into the 
Generis software as the default. As stated that further data acquisition will take place, 
since then data has been acquired from twenty more subjects, distributed almost equally between 
patients and controls.\\
\\
To ensure ease of interpretation of arithmetic cues and instructions, the default time for cue 
delivery in the arithmetic phase of \gls{daq} has been increased by 60s and audio recordings of 
the cues and instructions has ben incroporated into the Generis software for all major Nigerian 
languages.\\
\\
A handheld annotator is currently under design to take subject and clinician feedback on comfort, 
artifact activity and arithmetic task completion.\\
\\In trying to establish a best data processing path, certain areas of possible improvements were 
noted and acted on. One of such is the processing of the auditory phase signals from which the \gls{mmn} 
waveform is computed. Previously,the three tone classes, one standard and two 
deviant were plotted directly. The \gls{mmn} waveform is meant to be computed as the difference between 
the deviants and the standard tone. This has been corrected and the results shown under section \ref{sec:figures}.
Also to further smoothen the MMN waveform, a five point moving average was used and a minimum-maximum 
scaler was used on each waveform to make visualization easier.\\
\\
Formerly, in computing the fuzzy entropy, the library used had a minimum space complexity rquirement. 
Electrode of cortical regions of high spatial proximity were combined to overcome this. This might 
lead to some information loss. This has been improved by augmenting electrode of each region with channels 
of gaussina noise od zero mean and a unit standard deviation. The new results are shown under 
section \ref{sec:figures}. Also a self developed fuzzy entropy algorithm not having this limitation, has 
been developed after extensive literature review on fuzzy entropy. The code is given in section 
\ref{sec:appendix}

\section{Next Steps}
Over the period of four weeks, I will be doing the following

\section{Appendix}\label{sec:appendix}

\section{Figures}\label{sec:figures}
\begin{figure}[H]
  \includegraphics[width=16cm]{../../../data_analysis_results/FuzzEnt/Control/all-fuzzyEntr.png}
  \caption{Fuzzy Entropy from controls}
  \label{fig:controlFuzzEnt}
\end{figure}

\begin{figure}[H]
  \includegraphics[width=16cm]{../../../data_analysis_results/FuzzEnt/Patient/all-fuzzyEntr.png}
  \caption{Fuzzy Entropy from patients}
  \label{fig:patientFuzzEnt}
\end{figure}

\begin{figure}[H]
  \includegraphics[width=16cm]{../../../data_analysis_results/MMN/time_series/Control/24.png}
  \caption{A control subject \gls{mmn} plots}
  \label{fig:controlMMN}
\end{figure}

\begin{figure}[H]
  \includegraphics[width=16cm]{../../../data_analysis_results/MMN/time_series/Patient/10.png}
  \caption{A \gls{sz} subject \gls{mmn} plots}
  \label{fig:patientMMN}
\end{figure}

% \begin{figure}[H]
%   \includegraphics{images/preProcessing.jpeg}
%   \caption{data preprocessing architecture}
%   \label{fig:data preprocessing}
% \end{figure}

% \begin{figure}[H]
%   \centering
%   \subfloat[\centering montage from 1-70hz bandpass filtering]{{\includegraphics[width=5cm]{images/ordinary_montage.png}}}
%   \subfloat[\centering montage from filtering \& edge-interpolation]{{\includegraphics[width=5cm]{images/edge_mean_montage.png}}}
%   \subfloat[\centering montage from filtering, edge-interpolation \& baseline-correction]{{\includegraphics[width=5cm]{images/edge_mean_baseline_montage.png}}}
%   \caption{montage plots, first outlook}
%   \label{fig:montages}
% \end{figure}
% \begin{figure}[H]
%   \includegraphics{images/simiat_temporal1.png}
%   \caption{plot of tone averages from auditory stimli phase during 1st outlook of data}
%   \label{fig:temporal outlook}
% \end{figure}

% \begin{landscape}
%   \begin{figure}
%     \includegraphics[width=7.2in]{images/control.png}
%     \caption{feature computation from \gls{hc}}
%     \label{control features}
%   \end{figure}
% \end{landscape}
% \begin{landscape}
%   \begin{figure}
%     \includegraphics[width=7.2in]{images/patient.png}
%     \caption{feature computation from \gls{szPtnt}}
%     \label{patient features}
%   \end{figure}
% \end{landscape}

% \begin{figure}[H]
%   \centering
%   \includegraphics[width=13cm]{images/daqScreen.png}
%   \caption{montage plots, first outlook}
%   \label{fig:acquisitionScreen}
% \end{figure}
% \begin{figure}[H]
%   \centering
%   \includegraphics[width=8cm]{images/SessionParamsScreen.png}
%   \caption{Confifuring session settings}
%   \label{fig:paramsSreen}
% \end{figure}
% \begin{figure}[H]
%   \centering
%   \includegraphics{images/CreateSubjectScreen.resized.png}
%   \caption{Create subject screen}
%   \label{fig:subjectCreate}
% \end{figure}

% \begin{figure}[H]
%   \rotatebox[origin=c]{90}{\includegraphics{images/kt1018 flowchart.jpeg}}
%   % \includegraphics{images/kt1018 flowchart.jpeg}
%   \caption{acquiring one sample of eeg data from }
%   \label{fig:kt108 sample}
% \end{figure}

\end{document}
% \include{monthlly_report.acn}

%\journal{CSc 4110 Final Report}

%\title[journalExample]{Format for Project Reports}
\title{
  An update on the project: 
  \textbf{
      \textit{
        Development of an Automatic Instrument for Schizophrenia(SZ) Diagnosis
        }
      }, for the MCIP Innovation Prize 2022.
  }
% \author{
% Emmanuel OLATEJU \\
%     \begin{affiliation}
%       Supervised by Dr. K.P. Ayodele \\ 14/02/2023, \\
%       email: \mbox{kayodele@gmail.com, eoolateju@student.oauife.edu.ng}
%     \end{affiliation}
% }

\begin{document}
\maketitle

\section{Summary}
The previous report presented the data acquisition and processing methods along some obtained 
results. Also highlighted were technical issues noticed during data acquisition and 
issues with ergonomics of developed software for the clinicians. The previous report also 
presented the next steps to be taken.\\
\\
This report will highlight the challenges and next steps stated within the previous eport, then 
then present the progress made from the time of submission of the last report till the moment of 
submission of this report, more recent challenges and the next steps to be taken.

\section{From February's Report}
The last report presented results on some extracted features and stated some challenges the project is 
facing which include scheduling and mobility issues, 
subject attitude towards participation in exercise, cue communication and dellivery method, absence of 
clinician, subject feedback system, fuzzy entropy spatial complexity algoritm problems, non-uniform session times.\\
\\
Also stated within the last report were the set of steps to be taken next. This steps included establishing 
the best pre-processing path in terms of features being more discriminable, resolving left over issues 
with audio cue delivery mechanism, development of hand held annotator for taking feedback from clinician and 
subject, developing montage analysis algorithm, improvement of data acquisition paradigm and more data acquisition.\\
\\
In the next section, what has been achieved is discussed.

\section{Progress}
In order to establish uniform data acquisition paradigm so that processing of data can be made 
easier, from the already acquired data, the minumum permissible timeframe that can be met from 
the sessions data of all subjects was adopted for data processing and incroporated into the 
Generis software as the default. As stated that further data acquisition will take place, 
since then data has been acquired from twenty more subjects, distributed almost equally between 
patients and controls.\\
\\
To ensure ease of interpretation of arithmetic cues and instructions, the default time for cue 
delivery in the arithmetic phase of \gls{daq} has been increased by 60s and audio recordings of 
the cues and instructions has ben incroporated into the Generis software for all major Nigerian 
languages.\\
\\
A handheld annotator is currently under design to take subject and clinician feedback on comfort, 
artifact activity and arithmetic task completion.\\
\\In trying to establish a best data processing path, certain areas of possible improvements were 
noted and acted on. One of such is the processing of the auditory phase signals from which the \gls{mmn} 
waveform is computed. Previously,the three tone classes, one standard and two 
deviant were plotted directly. The \gls{mmn} waveform is meant to be computed as the difference between 
the deviants and the standard tone. This has been corrected and the results shown under section \ref{sec:figures}.
Also to further smoothen the MMN waveform, a five point moving average was used and a minimum-maximum 
scaler was used on each waveform to make visualization easier.\\
\\
Formerly, in computing the fuzzy entropy, the library used had a minimum space complexity rquirement. 
Electrode of cortical regions of high spatial proximity were combined to overcome this. This might 
lead to some information loss. This has been improved by augmenting electrode of each region with channels 
of gaussina noise od zero mean and a unit standard deviation. The new results are shown under 
section \ref{sec:figures}. Also a self developed fuzzy entropy algorithm not having this limitation, has 
been developed after extensive literature review on fuzzy entropy. The code is given in section 
\ref{sec:appendix}

\section{Next Steps}
Over the period of four weeks, I will be doing the following

\section{Appendix}\label{sec:appendix}

\section{Figures}\label{sec:figures}
\begin{figure}[H]
  \includegraphics[width=16cm]{../../../data_analysis_results/FuzzEnt/Control/all-fuzzyEntr.png}
  \caption{Fuzzy Entropy from controls}
  \label{fig:controlFuzzEnt}
\end{figure}

\begin{figure}[H]
  \includegraphics[width=16cm]{../../../data_analysis_results/FuzzEnt/Patient/all-fuzzyEntr.png}
  \caption{Fuzzy Entropy from patients}
  \label{fig:patientFuzzEnt}
\end{figure}

\begin{figure}[H]
  \includegraphics[width=16cm]{../../../data_analysis_results/MMN/time_series/Control/24.png}
  \caption{A control subject \gls{mmn} plots}
  \label{fig:controlMMN}
\end{figure}

\begin{figure}[H]
  \includegraphics[width=16cm]{../../../data_analysis_results/MMN/time_series/Patient/10.png}
  \caption{A \gls{sz} subject \gls{mmn} plots}
  \label{fig:patientMMN}
\end{figure}

% \begin{figure}[H]
%   \includegraphics{images/preProcessing.jpeg}
%   \caption{data preprocessing architecture}
%   \label{fig:data preprocessing}
% \end{figure}

% \begin{figure}[H]
%   \centering
%   \subfloat[\centering montage from 1-70hz bandpass filtering]{{\includegraphics[width=5cm]{images/ordinary_montage.png}}}
%   \subfloat[\centering montage from filtering \& edge-interpolation]{{\includegraphics[width=5cm]{images/edge_mean_montage.png}}}
%   \subfloat[\centering montage from filtering, edge-interpolation \& baseline-correction]{{\includegraphics[width=5cm]{images/edge_mean_baseline_montage.png}}}
%   \caption{montage plots, first outlook}
%   \label{fig:montages}
% \end{figure}
% \begin{figure}[H]
%   \includegraphics{images/simiat_temporal1.png}
%   \caption{plot of tone averages from auditory stimli phase during 1st outlook of data}
%   \label{fig:temporal outlook}
% \end{figure}

% \begin{landscape}
%   \begin{figure}
%     \includegraphics[width=7.2in]{images/control.png}
%     \caption{feature computation from \gls{hc}}
%     \label{control features}
%   \end{figure}
% \end{landscape}
% \begin{landscape}
%   \begin{figure}
%     \includegraphics[width=7.2in]{images/patient.png}
%     \caption{feature computation from \gls{szPtnt}}
%     \label{patient features}
%   \end{figure}
% \end{landscape}

% \begin{figure}[H]
%   \centering
%   \includegraphics[width=13cm]{images/daqScreen.png}
%   \caption{montage plots, first outlook}
%   \label{fig:acquisitionScreen}
% \end{figure}
% \begin{figure}[H]
%   \centering
%   \includegraphics[width=8cm]{images/SessionParamsScreen.png}
%   \caption{Confifuring session settings}
%   \label{fig:paramsSreen}
% \end{figure}
% \begin{figure}[H]
%   \centering
%   \includegraphics{images/CreateSubjectScreen.resized.png}
%   \caption{Create subject screen}
%   \label{fig:subjectCreate}
% \end{figure}

% \begin{figure}[H]
%   \rotatebox[origin=c]{90}{\includegraphics{images/kt1018 flowchart.jpeg}}
%   % \includegraphics{images/kt1018 flowchart.jpeg}
%   \caption{acquiring one sample of eeg data from }
%   \label{fig:kt108 sample}
% \end{figure}

\end{document}
% \include{monthlly_report.acn}

%\journal{CSc 4110 Final Report}

%\title[journalExample]{Format for Project Reports}
\title{
  An update on the project: 
  \textbf{
      \textit{
        Development of an Automatic Instrument for Schizophrenia(SZ) Diagnosis
        }
      }, for the MCIP Innovation Prize 2022.
  }
% \author{
% Emmanuel OLATEJU \\
%     \begin{affiliation}
%       Supervised by Dr. K.P. Ayodele \\ 14/02/2023, \\
%       email: \mbox{kayodele@gmail.com, eoolateju@student.oauife.edu.ng}
%     \end{affiliation}
% }

\begin{document}
\maketitle

\section{Summary}
The previous report presented the data acquisition and processing methods along some obtained 
results. Also highlighted were technical issues noticed during data acquisition and 
issues with ergonomics of developed software for the clinicians. The previous report also 
presented the next steps to be taken.\\
\\
This report will highlight the challenges and next steps stated within the previous eport, then 
then present the progress made from the time of submission of the last report till the moment of 
submission of this report, more recent challenges and the next steps to be taken.

\section{From February's Report}
The last report presented results on some extracted features and stated some challenges the project is 
facing which include scheduling and mobility issues, 
subject attitude towards participation in exercise, cue communication and dellivery method, absence of 
clinician, subject feedback system, fuzzy entropy spatial complexity algoritm problems, non-uniform session times.\\
\\
Also stated within the last report were the set of steps to be taken next. This steps included establishing 
the best pre-processing path in terms of features being more discriminable, resolving left over issues 
with audio cue delivery mechanism, development of hand held annotator for taking feedback from clinician and 
subject, developing montage analysis algorithm, improvement of data acquisition paradigm and more data acquisition.\\
\\
In the next section, what has been achieved is discussed.

\section{Progress}
In order to establish uniform data acquisition paradigm so that processing of data can be made 
easier, from the already acquired data, the minumum permissible timeframe that can be met from 
the sessions data of all subjects was adopted for data processing and incroporated into the 
Generis software as the default. As stated that further data acquisition will take place, 
since then data has been acquired from twenty more subjects, distributed almost equally between 
patients and controls.\\
\\
To ensure ease of interpretation of arithmetic cues and instructions, the default time for cue 
delivery in the arithmetic phase of \gls{daq} has been increased by 60s and audio recordings of 
the cues and instructions has ben incroporated into the Generis software for all major Nigerian 
languages.\\
\\
A handheld annotator is currently under design to take subject and clinician feedback on comfort, 
artifact activity and arithmetic task completion.\\
\\In trying to establish a best data processing path, certain areas of possible improvements were 
noted and acted on. One of such is the processing of the auditory phase signals from which the \gls{mmn} 
waveform is computed. Previously,the three tone classes, one standard and two 
deviant were plotted directly. The \gls{mmn} waveform is meant to be computed as the difference between 
the deviants and the standard tone. This has been corrected and the results shown under section \ref{sec:figures}.
Also to further smoothen the MMN waveform, a five point moving average was used and a minimum-maximum 
scaler was used on each waveform to make visualization easier.\\
\\
Formerly, in computing the fuzzy entropy, the library used had a minimum space complexity rquirement. 
Electrode of cortical regions of high spatial proximity were combined to overcome this. This might 
lead to some information loss. This has been improved by augmenting electrode of each region with channels 
of gaussina noise od zero mean and a unit standard deviation. The new results are shown under 
section \ref{sec:figures}. Also a self developed fuzzy entropy algorithm not having this limitation, has 
been developed after extensive literature review on fuzzy entropy. The code is given in section 
\ref{sec:appendix}

\section{Next Steps}
Over the period of four weeks, I will be doing the following

\section{Appendix}\label{sec:appendix}

\section{Figures}\label{sec:figures}
\begin{figure}[H]
  \includegraphics[width=16cm]{../../../data_analysis_results/FuzzEnt/Control/all-fuzzyEntr.png}
  \caption{Fuzzy Entropy from controls}
  \label{fig:controlFuzzEnt}
\end{figure}

\begin{figure}[H]
  \includegraphics[width=16cm]{../../../data_analysis_results/FuzzEnt/Patient/all-fuzzyEntr.png}
  \caption{Fuzzy Entropy from patients}
  \label{fig:patientFuzzEnt}
\end{figure}

\begin{figure}[H]
  \includegraphics[width=16cm]{../../../data_analysis_results/MMN/time_series/Control/24.png}
  \caption{A control subject \gls{mmn} plots}
  \label{fig:controlMMN}
\end{figure}

\begin{figure}[H]
  \includegraphics[width=16cm]{../../../data_analysis_results/MMN/time_series/Patient/10.png}
  \caption{A \gls{sz} subject \gls{mmn} plots}
  \label{fig:patientMMN}
\end{figure}

% \begin{figure}[H]
%   \includegraphics{images/preProcessing.jpeg}
%   \caption{data preprocessing architecture}
%   \label{fig:data preprocessing}
% \end{figure}

% \begin{figure}[H]
%   \centering
%   \subfloat[\centering montage from 1-70hz bandpass filtering]{{\includegraphics[width=5cm]{images/ordinary_montage.png}}}
%   \subfloat[\centering montage from filtering \& edge-interpolation]{{\includegraphics[width=5cm]{images/edge_mean_montage.png}}}
%   \subfloat[\centering montage from filtering, edge-interpolation \& baseline-correction]{{\includegraphics[width=5cm]{images/edge_mean_baseline_montage.png}}}
%   \caption{montage plots, first outlook}
%   \label{fig:montages}
% \end{figure}
% \begin{figure}[H]
%   \includegraphics{images/simiat_temporal1.png}
%   \caption{plot of tone averages from auditory stimli phase during 1st outlook of data}
%   \label{fig:temporal outlook}
% \end{figure}

% \begin{landscape}
%   \begin{figure}
%     \includegraphics[width=7.2in]{images/control.png}
%     \caption{feature computation from \gls{hc}}
%     \label{control features}
%   \end{figure}
% \end{landscape}
% \begin{landscape}
%   \begin{figure}
%     \includegraphics[width=7.2in]{images/patient.png}
%     \caption{feature computation from \gls{szPtnt}}
%     \label{patient features}
%   \end{figure}
% \end{landscape}

% \begin{figure}[H]
%   \centering
%   \includegraphics[width=13cm]{images/daqScreen.png}
%   \caption{montage plots, first outlook}
%   \label{fig:acquisitionScreen}
% \end{figure}
% \begin{figure}[H]
%   \centering
%   \includegraphics[width=8cm]{images/SessionParamsScreen.png}
%   \caption{Confifuring session settings}
%   \label{fig:paramsSreen}
% \end{figure}
% \begin{figure}[H]
%   \centering
%   \includegraphics{images/CreateSubjectScreen.resized.png}
%   \caption{Create subject screen}
%   \label{fig:subjectCreate}
% \end{figure}

% \begin{figure}[H]
%   \rotatebox[origin=c]{90}{\includegraphics{images/kt1018 flowchart.jpeg}}
%   % \includegraphics{images/kt1018 flowchart.jpeg}
%   \caption{acquiring one sample of eeg data from }
%   \label{fig:kt108 sample}
% \end{figure}

\end{document}
% \include{monthlly_report.acn}

%\journal{CSc 4110 Final Report}

%\title[journalExample]{Format for Project Reports}
\title{
  An update on the project: 
  \textbf{
      \textit{
        Development of an Automatic Instrument for Schizophrenia(SZ) Diagnosis
        }
      }, for the MCIP Innovation Prize 2022.
  }
% \author{
% Emmanuel OLATEJU \\
%     \begin{affiliation}
%       Supervised by Dr. K.P. Ayodele \\ 14/02/2023, \\
%       email: \mbox{kayodele@gmail.com, eoolateju@student.oauife.edu.ng}
%     \end{affiliation}
% }

\begin{document}
\maketitle

\section{Summary}
The previous report presented the fuzzy entropy features extracted from cortical
regions during phases of \gls{daq} and also the refined \gls{mmn} plots which 
made the \gls{mmn} waveform more evident. Some of the next steps stated were
\begin{itemize}
  \item The continued iprovement of fuzzy-entropy library to work with multivariate \\
  time-series(2D data)
  \item comparing performance of the fuzzy-entropy features from sourced library to \\
  that of the personally developed library
  \item Computing the auditory steady state features
\end{itemize}
A detour was made from the intended steps to be taken as stated within the previous report.
The extracted features have been analysed to get a sense of how discriminable the 
subject classes are using these features so as to implement the classifier instrument. 
The \gls{mmn} features were found to offer some level of discriminability in the 
temporal lobe and frontal lobe, while the fuzzy entropy feature was not discriminable, 
but correlation analysis showed that the fuzzy entropy might contain some useful 
information.

\section{From March's Report}
In the previous report(for march submitted on the 3rd of April), the following was highlighted:
\begin{itemize}
  \item Recomputing the \gls{mmn} waveforms
  \item Spatial dimension augmentation with gaussian noise during computation of fuzzy-entropy
  \item Development of custom fuzzy-entropy library
  \item Improvement in ergonomics of software for subject by implementing auditory cues for all 
  major Nigerian languages
  \item Development of an hand held annotator device to improve ergonomics for clinician and subject
\end{itemize} 

\section{Progress}
As stated, a detour from the intended next steps from the last report was taken. 
Analysis of the extracted features was done so as to justify upcoming changes in 
features computation. The results of the analysis are given in 
section \ref{figures}. The set of analysis done includes:
\begin{itemize}
  \item Computation of \gls{mmn} value from \gls{mmn} waveform in 0-100ms, 100-200ms, 
  200-300ms, 300-400ms, 400-450ms windows.(Code \ref{mmn_value_code})
  \item Comparing \gls{mmn} values for 1KHz duration deviant and 3KHz frequency deviant 
  between patient and controls across all electrodes and time windows.
  (Figures \ref{mmnvalue_0_100ms} - \ref{mmnvalue_400_450ms})
  \item Re-implementing montage plotting code to improve time-complexity.(Code \ref{montage_plot_code})
  \item Converting the \gls{mmn} values to montage plots for 1KHz duration deviant 
  and 3KHz frequency deviant for easier interpretation.
  (Figures \ref{control_1KHz_mmn_montage}-\ref{patient_3KHz_mmn_montage})
  \item Comparing distribution of computed fuzzy-entropy values of patients and controls 
  for each cortical region across all phases of \gls{daq}.
  (Figures \ref{fig:controlFuzzEnt}-\ref{fuzz_ent_distributions}) 
  \item Correlating fuzzy-entropy values from cortical regions across 
  all phases of \gls{daq}.(Figure \ref{fuzz_ent_corr_mat})
  \item Comparing entropies from all cortical regions for each phase of \gls{daq}.
  (Figures \ref{rest_fuzz}-\ref{auditory_fuzz})
  \item Comparing entropies from all phases of \gls{daq} for each cortical region.
  (figures \ref{frontal_fuzz}-\ref{temporal_fuzz})
\end{itemize} 

\section{Inference}
The following infernce are drawn:
\begin{itemize}
  \item The \gls{mmn} features in the 1KHz ad 3KHz deviant are more 
  evident in different cortical regions. 
  \item Using the \gls{mmn} features of the temporal lobe 
  from the two tone deviants, the \glspl{hc} can be discriminated 
  from the \glspl{szPtnt}
  \item The fuzzy entropy features must be recomputed or undergo 
  further processing to provide discriminable information
  \item The fuzzy entropy of the frontal lobes might contain information 
  that is discriminable between both classes. It exhibits different 
  correlation patterns from the other cortical regions. This occurs to 
  a less significant extent in the temporal and occipital lobes.
\end{itemize}

\section{Next Steps}
In this month of may, I will be doing the following
\begin{itemize}
  \item Computing the auditory steady state features
  \item Re-computing fuzzy-entropy features using other libraries and 
  comparing performance.(Code \ref{dev_fuzzyentropy_code})
  \item Improve discriminability of features using spatial filters.
  \item Apply all features to a custom mutilayer feedforward network 
  for classification.
\end{itemize}

\clearpage
\section{Figures}\label{figures}
\begin{figure}[H]
  \includegraphics[width=16cm]{../../../data_analysis_results/MMN/features/deviant_tone_0.png}
  \caption{\gls{mmn} values from 0-100ms}\label{mmnvalue_0_100ms}
\end{figure}
\begin{figure}[H]
  \includegraphics[width=16cm]{../../../data_analysis_results/MMN/features/deviant_tone_1.png}
  \caption{\gls{mmn} values from 100-200ms}\label{mmnvalue_100_200ms}
\end{figure}
\begin{figure}[H]
  \includegraphics[width=16cm]{../../../data_analysis_results/MMN/features/deviant_tone_2.png}
  \caption{\gls{mmn} values from 200-300ms}\label{mmnvalue_200_300ms}
\end{figure}
\begin{figure}[H]
  \includegraphics[width=16cm]{../../../data_analysis_results/MMN/features/deviant_tone_3.png}
  \caption{\gls{mmn} values from 300-400ms}\label{mmnvalue_300_400ms}
\end{figure}
\begin{figure}[H]
  \includegraphics[width=16cm]{../../../data_analysis_results/MMN/features/deviant_tone_4.png}
  \caption{\gls{mmn} values from 400-450ms}\label{mmnvalue_400_450ms}
\end{figure}

\begin{figure}[H]
  \includegraphics[width=16cm]{../../../data_analysis_results/MMN/montage/Control/1KHz_duration_deviant_montage.png}
  \caption{Controls 1KHz duration deviant \gls{mmn} value montage}\label{control_1KHz_mmn_montage}
\end{figure}
\begin{figure}[H]
  \includegraphics[width=16cm]{../../../data_analysis_results/MMN/montage/Patient/1KHz_duration_deviant_montage.png}
  \caption{Patients 1KHz duration deviant \gls{mmn} value montage}\label{patient_1KHz_mmn_montage}
\end{figure}
\begin{figure}[H]
  \includegraphics[width=16cm]{../../../data_analysis_results/MMN/montage/Control/3KHz_frequency_deviant_montage.png}
  \caption{Controls 3KHz frequency deviant \gls{mmn} value montage}\label{control_3KHz_mmn_montage}
\end{figure}
\begin{figure}[H]
  \includegraphics[width=16cm]{../../../data_analysis_results/MMN/montage/Patient/3KHz_frequency_deviant_montage.png}
  \caption{Patients 3KHz frequency deviant \gls{mmn} value montage}\label{patient_3KHz_mmn_montage}
\end{figure}

\begin{figure}[H]
  \includegraphics[width=16cm]{../../../data_analysis_results/FuzzEnt/Control/all-fuzzyEntr.png}
  \caption{Fuzzy Entropy from controls}\label{fig:controlFuzzEnt}
\end{figure}
\begin{figure}[H]
  \includegraphics[width=16cm]{../../../data_analysis_results/FuzzEnt/Patient/all-fuzzyEntr.png}
  \caption{Fuzzy Entropy from patients}\label{fig:patientFuzzEnt}
\end{figure}
\begin{figure}[H]
  \includegraphics[width=16cm]{../../../data_analysis_results/FuzzEnt/corticalRegions_DAQphase_distributions.png}
  \caption{Fuzzy Entropy from controls}\label{fuzz_ent_distributions}
\end{figure}

\begin{figure}[H]
  \includegraphics[width=16cm]{../../../data_analysis_results/FuzzEnt/entropies_corr_mat.png}
  \caption{Fuzzy-entropy values correlation smatrix}\label{fuzz_ent_corr_mat}
\end{figure}

\begin{figure}[H]
  \includegraphics[width=16cm]{../../../data_analysis_results/FuzzEnt/rest_phases_corr.png}
  \caption{Rest Phases Fuzzy-entropy}\label{rest_fuzz}
\end{figure}
\begin{figure}[H]
  \includegraphics[width=16cm]{../../../data_analysis_results/FuzzEnt/arith_phases_corr.png}
  \caption{Arithmetic Phase Fuzzy-entropy}\label{arith_fuzz}
\end{figure}
\begin{figure}[H]
  \includegraphics[width=16cm]{../../../data_analysis_results/FuzzEnt/auditory_phases_corr.png}
  \caption{Auditory Phase Fuzzy-entropy}\label{auditory_fuzz}
\end{figure}

\begin{figure}[H]
  \includegraphics[width=16cm]{../../../data_analysis_results/FuzzEnt/frontal_region_corr.png}
  \caption{Frontal lobe Fuzzy-entropy}\label{frontal_fuzz}
\end{figure}
\begin{figure}[H]
  \includegraphics[width=16cm]{../../../data_analysis_results/FuzzEnt/parietal_region_corr.png}
  \caption{Parietal lobe Fuzzy-entropy}\label{parietal_fuzz}
\end{figure}
\begin{figure}[H]
  \includegraphics[width=16cm]{../../../data_analysis_results/FuzzEnt/central_region_corr.png}
  \caption{Central lobe Fuzzy-entropy}\label{central_fuzz}
\end{figure}
\begin{figure}[H]
  \includegraphics[width=16cm]{../../../data_analysis_results/FuzzEnt/occipital_region_corr.png}
  \caption{Occipital lobe Fuzzy-entropy}\label{occipital_fuzz}
\end{figure}
\begin{figure}[H]
  \includegraphics[width=16cm]{../../../data_analysis_results/FuzzEnt/temporal_region_corr.png}
  \caption{Temporal lobe Fuzzy-entropy}\label{temporal_fuzz}
\end{figure}

\clearpage
\section{Appendix}\label{sec:appendix}
\subsection{\gls{mmn} time window value code}\label{mmn_value_code}
\begin{lstlisting}
import numpy as np

class MMN:
    def __init__(self,delay=10,position=40,window_size=20):
        self.d = delay
        self.p = position
        self.w = window_size
    
    def mmn_value(self,x):
        if x.ndim == 1:
            return np.mean(x[self.p-self.d:self.p-self.d+self.w])
        elif x.ndim == 2:
            return np.mean(x[:,self.p-self.d:self.p-self.d+self.w],axis=1)
\end{lstlisting}

\subsection{montage plot code}\label{montage_plot_code}
\begin{lstlisting}
  import numpy as np
  from matplotlib import pyplot as plt, patches
  import mne
  
  from scipy.interpolate import PchipInterpolator,interp1d
  
  import os
  
  from math import cos, pi, sin, radians
  from scipy.interpolate import griddata,interp2d
  
  r1 = 0.1
  r2 =0.13
  center = (r2+0.01,r2+0.01)
  
  
  ELECTRODES = {
      'FpZ':(r2+0.01+(r1*cos(radians(90))),r2+0.01+(r1*sin(radians(90)))),
      'OZ':(r2+0.01+(r1*cos(radians(270))),r2+0.01+(r1*sin(radians(270)))),
      'PZ':(r2+0.01+(0.48*r1*cos(radians(270))),r2+0.01+(0.48*r1*sin(radians(270)))),
      'FZ':(r2+0.01+(0.48*r1*cos(radians(90))),r2+0.01+(0.48*r1*sin(radians(90)))),
      'CZ':(r2+0.01,r2+0.01),
  
      'Fp1':(r2+0.01+(0.95*r1*cos(radians(100))),r2+0.01+(0.95*r1*sin(radians(100)))),
      'Fp2':(r2+0.01+(0.95*r1*cos(radians(80))),r2+0.01+(0.95*r1*sin(radians(80)))),
      'F7':(r2+0.01+(0.95*r1*cos(radians(145))),r2+0.01+(0.95*r1*sin(radians(145)))),
      'F8':(r2+0.01+(0.95*r1*cos(radians(35))),r2+0.01+(0.95*r1*sin(radians(35)))),
      'T7':(r2+0.01+(r1*cos(radians(180))),r2+0.01+(r1*sin(radians(180)))),
      'T8':(r2+0.01+(r1*cos(radians(0))),r2+0.01+(r1*sin(radians(0)))),
      'P7':(r2+0.01+(r1*cos(radians(215))),r2+0.01+(r1*sin(radians(215)))),
      'P8':(r2+0.01+(r1*cos(radians(-35))),r2+0.01+(r1*sin(radians(-35)))),
      'O1':(r2+0.01+(0.35*r1*cos(radians(240))),r2+0.01+(0.95*r1*sin(radians(240)))),
      'O2':(r2+0.01+(0.35*r1*cos(radians(-60))),r2+0.01+(0.95*r1*sin(radians(-60)))),
  
      'F3':(r2+0.01+(0.48*r1*cos(radians(90)))-(0.4*r1),r2+0.01+(0.48*r1*sin(radians(90)))),
      'F4':(r2+0.01+(0.48*r1*cos(radians(90)))+(0.4*r1),r2+0.01+(0.48*r1*sin(radians(90)))),
      'P3':(r2+0.01+(0.48*r1*cos(radians(270)))-(0.4*r1),r2+0.01+(0.48*r1*sin(radians(270)))),
      'P4':(r2+0.01+(0.48*r1*cos(radians(270)))+(0.4*r1),r2+0.01+(0.48*r1*sin(radians(270)))),
      'C3':(r2+0.01+(0.5*r1*cos(radians(180))),r2+0.01+(0.5*r1*sin(radians(180)))),
      'C4':(r2+0.01+(0.5*r1*cos(radians(0))),r2+0.01+(0.5*r1*sin(radians(0)))),
      'T3':(r2+0.01+(0.95*r1*cos(radians(180))),r2+0.01+(0.95*r1*sin(radians(180)))),
      'T4':(r2+0.01+(0.95*r1*cos(radians(0))),r2+0.01+(0.95*r1*sin(radians(0)))),
      'T5':(r2+0.01+(0.95*r1*cos(radians(220))),r2+0.01+(0.95*r1*sin(radians(220)))),
      'T6':(r2+0.01+(0.95*r1*cos(radians(-40))),r2+0.01+(0.95*r1*sin(radians(-40)))),
  }
  
  EDGE_COORDS = [
      (r2+0.01+(r2*cos(radians(90))),r2+0.01+(r2*sin(radians(90)))),
      (r2+0.01+(r2*cos(radians(270))),r2+0.01+(r2*sin(radians(270)))),
      (r2+0.01+(r2*cos(radians(115))),r2+0.01+(r2*sin(radians(115)))),
      (r2+0.01+(r2*cos(radians(65))),r2+0.01+(r2*sin(radians(65)))),
      (r2+0.01+(r2*cos(radians(145))),r2+0.01+(r2*sin(radians(145)))),
      (r2+0.01+(r2*cos(radians(35))),r2+0.01+(r2*sin(radians(35)))),
      (r2+0.01+(r2*cos(radians(180))),r2+0.01+(r2*sin(radians(180)))),
      (r2+0.01+(r2*cos(radians(0))),r2+0.01+(r2*sin(radians(0)))),
      (r2+0.01+(r2*cos(radians(215))),r2+0.01+(r2*sin(radians(215)))),
      (r2+0.01+(r2*cos(radians(-35))),r2+0.01+(r2*sin(radians(-35)))),
      (r2+0.01+(r2*cos(radians(245))),r2+0.01+(r2*sin(radians(245)))),
      (r2+0.01+(r2*cos(radians(-65))),r2+0.01+(r2*sin(radians(-65)))),
  ]
  
  def draw_electrode(electrodes,ax):
      for electrode in electrodes:
          position = ELECTRODES[electrode]
          ax.text(position[0], position[1], electrode,
              verticalalignment='center',
              horizontalalignment='center',
              size=10
          )
  
  def draw_nose(ax):
          """Draw nose."""
          nose1 = plt.Line2D([0.13,0.14],
                            [0.27,0.29],
                            linewidth=1,
                            color=(0, 0, 0))
          nose2 = plt.Line2D([0.14,0.15],
                            [0.29,0.27],
                            linewidth=1,
                            color=(0, 0, 0))
          ax.add_line(nose1)
          ax.add_line(nose2)
  
  def montage_plot(eeg_sample,electrodes,ax):
      head_outer_circle = patches.Circle(center, radius=r2+0.001, color='black')
      head_inner_circle = patches.Circle(center, radius=r2, color='white')
  
      # ax.set_aspect(1)
      ax.add_artist(head_outer_circle)
      ax.add_artist(head_inner_circle)
  
      draw_nose(ax)
      draw_electrode(electrodes,ax)
  
      points=[]
      for electrode in electrodes:
          points.append(ELECTRODES[electrode])
      points = points + EDGE_COORDS
      X,Y=np.mgrid[0:1:50j, 0:1:50j]
      Xi,Yi=np.meshgrid(X,Y)
      points = np.array(points)
  
      mu = np.mean(eeg_sample,0)
      mu_add = np.ones((len(EDGE_COORDS)))*mu
      data_sample = np.hstack((eeg_sample,mu_add))
      # data_sample = eeg_sample
  
      ax.set_ylim(0,2*(r2+0.02))
      ax.set_xlim(0,2*(r2+0.02))
  
      Z=griddata(points,data_sample,(Xi,Yi),'cubic')
      CS = ax.contour(Xi,Yi,Z,30,cmap='RdBu',extend='both',linewidths=20,extent=(0,1,0,1))
  
      # ax.axis('off')
      ax.get_xaxis().set_ticks([])
      ax.get_yaxis().set_ticks([])
      ax.patch.set_facecolor((1,1,1,0.01))
      ax.patch.set_alpha(0.0)
      # ax.set_aspect('equal')
      # plt.show()
\end{lstlisting}

\subsection{Fuzzy Entropy Code}\label{dev_fuzzyentropy_code}
\begin{lstlisting}
import numpy as np
from scipy.spatial.distance import cdist
import time

def sigmoid(x,r):
    assert isinstance(r,tuple), 'When Fx = "Sigmoid", r must be a two-element tuple.'
    y = 1/(1 + np.exp((x-r[1])/r[0]))
    return y  
def default(x,r):   
    assert isinstance(r,tuple), 'When Fx = "Default", r must be a two-element tuple.'
    y = np.exp(-(x**r[1])/r[0])
    return y     
def modsampen(x,r):
    assert isinstance(r,tuple), 'When Fx = "Modsampen", r must be a two-element tuple.'
    y = 1/(1 + np.exp((x-r[1])/r[0]))
    return y    
def gudermannian(x,r):
    if r <= 0:
        raise Exception('When Fx = "Gudermannian", r must be a scalar > 0.')
    y = np.arctan(np.tanh(r/x))    
    y = y/np.max(y)    
    return y    
def linear(x,r):    
    if r == 0 and x.shape[0]>1:    
        y = np.exp(-(x - np.min(x))/np.ptp(x))
    elif r == 1:
        y = np.exp(-(x - np.min(x)))
    elif r == 0 and x.shape[0]==1:   
        y = 0
    else:
        print(r)
        raise Exception('When Fx = "Linear", r must be 0 or 1')
    return y

class fuzzEntropy:
    def __init__(self,window_size,dissimilarity_index,membership_function='linear'):
        self.m = self.window_size = window_size
        self.r = self.dissimilarity_index = dissimilarity_index
        self.mu = self.membership_function = globals()[membership_function.lower()]
    
    def __computeFuzzyMatrix(self,x,m):
        if x.ndim == 1:
            N = x.shape[0]
            Xm = np.array([x[i:i+m-1].tolist() for i in range(0,N-m)])
            dm = cdist(Xm,Xm,'euclidean')
            dm = self.mu(dm,self.r)
            phim = np.sum(dm,axis=1)/(N-m+1)
        return phim
    
    def _fuzzyEntropyCompute(self,x):
        N = x.shape[0]
        phim = self.__computeFuzzyMatrix(x,self.m)
        phim1 = self.__computeFuzzyMatrix(x,self.m+1)

        psim = (1/(N-self.m+1)) * np.sum(phim,axis=0)
        psim1 = (1/(N-self.m+2)) * np.sum(phim1,axis=0)

        with np.errstate(divide='ignore', invalid='ignore'):
            fuzz = np.log(psim)-np.log(psim1)
        return fuzz
    
def fuzzEntropy2D(x,window_size,dissimilarity_index,membership_function=gaussianMembershipfunction):
        fuzzyent = fuzzEntropy(window_size,dissimilarity_index,membership_function)

        res = np.empty(x.shape[0])
        for i in range(x.shape[0]):
            res[i] = fuzzyent._fuzzyEntropyCompute(x[i,:])
        return res.mean()
\end{lstlisting}

\end{document}