\chapter{Introduction}\label{Ch:1}		% TODO: you can change the chapter title from Introduction if necessary

\section{Background}\label{sec:Intro}	
\ac{sz} is a brain disorder characterised by recurrent or continuous psychotic episodes (an altered view of reality) marked by 
auditory, visual, and delusional hallucinations, in addition to many other characteristics that are typical of \ac{sz} populations. 
Some other symptoms include consistent disorganised thinking, social withdrawal, decreased emotional expression and apathy. \ac{sz} is 
seen as an hetrogeneous syndrome and often referredto as a multitude of disease states\cite{bakhshi2015neuropathology}.

Statistics from 2011 indicate that there are 21 million people with \ac{sz} worldwide (about one of every 285). Data from the 
previous 50 years also demonstrates \ac{sz}'s comparatively steady occurrence throughout time. \ac{sz} is diagnosed in men 
1.4 times more frequently than in women, and it typically manifests in men earlier. Males typically experience psychosis at ages 20 to 
28 and females typically experience it at ages 26 to 32. Before the age of 13 and between the ages of 40 and 60, early and late onsets 
of psychosis are possible. The average age of patients being treated for \ac{sz} in hospitals is from 25 to 35 years old. The 
Middle East and East Asia have the highest populations of \ac{sz} worldwide, with the western and certain northern regions of 
Africa having the highest prevalence.

In its early phases, \ac{sz} is dormant and only gets worse with time. \ac{fep} is the pivotal point in the 
epidemiology of a \ac{sz} case; in the prodromal phase (pre-psychosis), it is initially asymptomatic (without noticeable symptoms), 
and it is only diagnosed after the \ac{fep}, which often occurs in adolescence or young adulthood. Positive, negative, or cognitive symptoms 
are all categories used to describe \ac{sz}. The cognitive symptoms are seen following \ac{fep} during mental tasks like arithmetic 
problems or difficult essay readings. The cognitive symptoms are less evident compared to the positive and negative symptoms in typical 
day-to-day activities. It can be challenging to make a diagnosis of \ac{sz} before \ac{fep} because these symptoms are frequently 
transitory and dormant before \ac{fep}.

Positive signs of psychosis include delusions, hallucinations, disorganised speech and thinking, and other types of altered responses and 
behaviour. Only after psychosis are the positive symptoms visible. 80\% of people with \ac{sz} experience hallucinations, which 
are largely related to auditory processes. The auditory hallucinations are a sign of a impaired auditory cortex (brain area responsible 
for auditory processes) functioning. The severity of hallucinations varies amongst individuals based on the number of sensory organs implicated and 
the degree of brain pathways linked to those sensory organs that are impaired. Passivity phenomenon, in which the patient feels that an 
outside force is influencing or controlling his or her thoughts and activities, is another helpful sign of \ac{sz}. Positive 
symptoms improve with treatment and lessen as the illness progresses.

Negative symptoms are impairments in regular emotional reactions or in other mental processes connected to feelings. Flat expressions or 
little emotion (blunt affect), poor speech (alogia), an inability to feel pleasure (asociality), a lack of desire to develop 
relationships (avolition), a lack of motivation, and apathy are some of the negative symptoms that have been discovered. Negative 
symptoms are rarely observed because they are indiscriminate and unique to each individual. Since they are tethered to human emotion, 
it is challenging to draw judgments about them only from their traits. Avolition and anhedonia are signs of a brain circuitry problem 
with reward processing which is associated with the prefrontal cortex of the brain. As a component of the brain's reward circuitry, the dorsal prefrontal cortex, brain signals from this region can shed light on how the reward circuitry differs in \ac{sz} populations and non \ac{sz} groups. Most \ac{sz} cases of negative symptoms are  associated with apathy and  are related to disrupted cognitive processing affecting memory and planning including goal directed-behaviour. 
Secondary negative symptoms are those that develop as a result of primary positive symptoms, antipsychotic drug side effects, substance 
use disorder, and social isolation, while primary negative symptoms are those that are innate to \ac{sz}. Compared to the original 
negative symptoms, the secondary negative symptoms are significantly more responsive to treatment. Negative symptoms are sometimes 
mistaken for hormonal fluctuations or stages of social disobedience, especially in adolescence, and are thus readily disregarded in 
day-to-day activities.

70\% of people with \ac{sz} experience cognitive symptoms, which are more pronounced and recurrent in both early and late stages 
of the illness. The existence and severity of cognitive dysfunction are seen as greater indicators of functionality than the presentation 
of core symptoms, despite the fact that they are not considered to be core symptoms of \ac{sz}. Cognitive deficiencies worsen 
during the initial psychotic episode, then recover to normal and stay largely stable throughout the disease. Social or non-social 
cognitive deficiencies are also possible. In populations with \ac{sz}, the following cognitive abilities are frequently accessed: verbal fluency, knowledge retention capacity, reasoning, problem-solving, processing speed, auditory perception, and visual perception. The most noticeable impairments are 
in verbal memory and attention, which all indicate damaged neural networks in the brain and altered brain function (reaction) to inputs. Cognitive impairment is also linked to episodic memory and visual backward masking. Antipsychotic medications have no effect on cognitive deficits; therapy is the preferred method of treatment.

Due to its difficult diagnosis, complex character, and concomitant post-psychotic behaviour symptomatic nature, which is mostly a 
combination of the behavioural symptoms of a significant class of mental diseases, \ac{sz} has been referred to as the heartland of 
psychiatry \cite{goodwin2007heartland}. \ac{sz} has no precise boundary spreading its psychotic manifestations across that of other 
mental disorders and its exact causation is unknown, for this reason, \ac{sz} is seen as an interplay of different factors causing 
multiple mental disorders occurring simultaneously and this also makes it difficult to identify \ac{sz} cases even after \ac{fep}. It 
is also hypothesised that it results from a complex interaction between genetic and environmental risk factors that affect early brain 
development and the course of biological adaptation to experiences in life. Of all hypothesised causative factors of \ac{sz}, the genetic and 
environmental factors have been more prevalent in cases of \ac{sz} and genetics hypothetically accounts for an estimated value of 
between 70\% and 80\% of \ac{sz} cases. However, \cite{etiologySZ} show that most people with \ac{sz} have no family history of psychosis. Also results of 
candidate gene studies have generally failed to find consistent associations. The question of how \ac{sz} could be primarily 
genetically influenced, given that \ac{sz} populations have lower fertility rates remains a paradox. However, consistently, the gene-loci 
explained by genome-wide association studies explains only a small fraction of the variations in \ac{sz}. This reduces the confidence level 
in using genetic factors in prediction of \ac{sz} remission, epidemiology of \ac{sz} and identification of \ac{sz} risk population.

Genetic and environmental biomarkers of \ac{sz} have produced inconsistent results in multiple works, thus existing methods of identifying \ac{sz} require some level of active psychosis so as to take advantage of positive and negative symptoms  and  generates behavioural information for psychiatric evaluation based on some psychiatric criteria such as the \ac{dsm} or the \ac{icd}. Very little empirical data such as \ac{mri} or \ac{ct} scan is used in identifying \ac{sz} and quality of psychiatric evaluation is subject to the experience and competence of the mental health official. Methods of treatment vary from patient to patient and after psychosis, is usually an individually tailored combination of talking therapy and medicine which is a very expensive process and mostly lifelong. \ac{sz} can be treated and managed if discovered early, before the onset of psychosis, but the pre-psychosis non salient nature of its symptoms make early identification, thus prevention difficult and the cost of treatment after \ac{fep} is high for the average worker accounting for the cost of therapy, antipsychotics, electroconvulsive therapy, hospitalisation and other processes. In the use of psychiatric evaluation methods the need of psychosis indicates already active \ac{sz} which is a mental condition that is best prevented as its treatment can be lifelong and similar talking therapy and medicine can prevent transition into psychosis. For these reasons, there is the need to develop a standard empirical test for identification of \ac{sz} risk population and active \ac{sz} patients.

Research works done on identifying the pathophysiology of \ac{sz} have produced results which are not generic to a significant percentage of the \ac{sz} population with inconsistencies in results employing the same approach. This may be due to the comorbid nature of mental-disorders in \ac{sz} patients, the variations in genetic material behaviour among patients, the dynamism of brain networks among patients and so many other factors. Also the symptoms of \ac{sz} can only be utilised after \ac{fep}. Research has not been able to establish a specific cause of the comorbid nature of \ac{sz} and thus has not been able to establish a particular factor as the primary agent responsible for \ac{sz}. Seeking for answers at the cell/tissue level of organisation seems to be a longshot, therefore, can higher levels of human biological organisation (organ/system) provide early diagnostic methods of \ac{sz} risk populations and \ac{sz} patients. This brings us to the study of the neuronal, brain and cortical structures of the brain, their characteristics, functions and how \ac{sz} alters their behaviour.

Based on the fact that prodromal and post-psychotic stages of \ac{sz} are consistently accompanied by an evolution (degradation) of brain information processing in schizophrenic populations, perhaps in developing an empirical test for \ac{sz} (characterising \ac{sz} patients and establishing links between genetic risk, brain biology, and other suspected factors), research should look beyond the symptoms and investigate the brain processes that result in these symptoms. As stated before, positive symptoms(hallucinations, delusions) indicate impaired auditory, visual and other sensory neural pathway functions in the brain, negative symptoms(apathy and social interaction) indicate impaired prefrontal cortex and other brain regions that act as seats of, or interact with emotional intelligence processing and also, cognitive symptoms indicate neural pathways responsible for cognitive functions are impaired some of which are found in the cerebral cortex and frontal cortex. Brain signals acquisition methods, such as the \ac{fep}, \ac{fmri}, \ac{fnirs}, are usually reflective of these changes. Research work done overtime has shown that some of these brain signal acquisition methods have given consistent results in discriminating between \ac{sz} patients, risk population and  non-\ac{sz} population. Some research also shows some of these successfully predict remission from \ac{sz} and predict the path taken by \ac{sz} towards or away from remission. Among the stated methods we will describe and focus on the non-invasive \ac{eeg}.
\subsection{Electroencephalography and Schizophtrenia}
The non-invasive \ac{eeg} is a method of acquiring electrical signals of the brain, which requires no surgical procedure. Signals from non-invasive \ac{eeg} are used to study the brain, understand its cortical level of organisation, study changes in cortical behaviour in response to events and in case of diseases and ailments. \ac{eeg} systems are preferred in most research works because of the balance between cost and quality of signal acquired in terms of temporal and spatial resolution. \ac{eeg} offers fine temporal resolution over spatial resolution, which is usually compensated for by the use of increased number of electrodes. In any study involving \ac{eeg} signals, certain signal classes are monitored in the frequency domain and certain time-domain behaviour is monitored. In the frequency domain, interested frequencies of oscillations are usually the beta(12-30Hz) for activities of motion and gamma(30Hz and above) for cognitive functions. Study of time-domain changes are usually attached to response to stimulus events, visual, auditory or somatosensory. Some time domain signal behaviours include \ac{p300}, \ac{n100}, \ac{mmn} and many more. And in certain cases the time-frequency domain gives information, a typical example being the \ac{ssvep}. \ac{eeg} based methods of identification of \ac{sz} patients and prediction of path of \ac{sz} evolution has provided consistent and statistically significant results over the years, which show that increased study of the \ac{eeg} signal classes in \ac{sz} and non \ac{sz} population will eventually lead to the development of a point of care, prognostic tool for \ac{sz}.

In light of the existence of auditory, visual and somatosensory hallucinations in \ac{sz} patients, impaired cognitive and social function, certain \ac{fep} signal classes that are indicative of the state of these functions have been found to be consistently altered in the \ac{sz} patients and risk population. One prominent class is the mismatch negativity which occurs in response to auditory and visual stimuli. The mismatch negativity (\ac{mmn}) is a component of the brain \ac{erp} to an odd stimulus in a sequence of stimuli. The \ac{mmn} signal occurs due to sudden change in parameters of the stimuli, such as frequency of sound within a standard time, time duration of signal of a standard frequency, sudden change in gradient of light intensity, etc. The \ac{mmn} signal is consistently attenuated in \ac{sz} at risk populations and \ac{sz} patients. Another class of \ac{fep} signal is the \ac{assr}. \ac{assr} is evoked using repeated sound stimuli presented at a high repetition rate and can be used to objectively estimate hearing sensitivity in individuals with normal hearing sensitivity and with various degrees and configurations of \ac{snhl}. \ac{sz} patients consistently show reduced \ac{assr} power and phase locking to gamma range stimulation. Since \ac{fep} results are significantly consistent in \ac{sz} patients and risk population,  how can they be used in developing a point of care prognostic measures for \ac{sz}.

\subsection{Problem Description}
In the quest of developing empirical methods of characterising \ac{sz} patients and discrimination \ac{sz} risk population from the non \ac{sz} populace, the medical and biological sciences seem to have hit a gridlock in seeking for pathological, environmental and genetic causes of \ac{sz} that consistently characterize \ac{sz} behaviour and its prognosis. In areas of genetics, inconsistencies have been found in the results of various research modalities. The comorbid nature of the \ac{sz} condition also makes it difficult to identify pathological causes of \ac{sz} and there exists limitations on social methods of determining environmental factors and causes of \ac{sz}, as these social methods may not be empirical enough to reach a conclusion.

\ac{eeg} methods of identifying \ac{sz} affected and risk population have provided more consistent results and has helped understand the ailment and develop various models that explained the ailment to an extent, one of such being the functional connectivity model which suggest reduced level of interaction between brain  regions. The functional connectivity model has shown that people with \ac{sz} show both higher diversity at each brain region and lower variance in connectivity strength across the brain. This can be conceptualised as a randomization or de-differentiation of functional connectivity.

\ac{eeg} methods have also suggested the use of auditory sensitive \ac{erp}'s in identifying \ac{sz} affected and risk populations. As stated before, the \ac{mmn} is a prominent \ac{erp} in \ac{fep} studies of \ac{sz}. The \ac{mmn} is consistently attenuated in \ac{sz} populace and has some other \ac{erp} components that are concomitant to it and altered in \ac{sz} patients. One such \ac{erp} is the P3a. Auditory P3a response is a fronto-centrally maximal positive component elicited by infrequent, unpredictable stimuli in a stream of repeating sounds and peaking between 200 and 400 ms from the stimulus onset. P3a is reduced in FES \ac{sz} patients and is usually indicative of attention and working memory. The \ac{mmn} is indicative of impaired auditory neural pathways, interpretation which can be due to deviation of attention patterns in the brain from the normal. Till date \ac{mmn} is the most stable identified marker of \ac{sz}, its temporally stable and heritable.

A temporal-frequency domain feature of \ac{eeg} whose behaviour is altered in \ac{sz} patients is the \ac{assr}. \ac{assr} demonstrates disturbances of neural synchrony and oscillations in \ac{sz} which affect a broad range of sensory and cognitive processes.These disturbances may account for a loss of neural integration and effective connectivity in the disorder. Interestingly, \ac{assr} has provided consistent evidence for lower levels of organisation defects in \ac{sz} populations. \ac{assr} may reflect disturbed interactions within \ac{gaba}ergic and glutamatergic circuits, particularly in the gamma range.

A persisting problem with use of \ac{eeg} methods, in most cases \ac{mmn} in discriminating \ac{sz} populace from the other populations is the existence of an overlap in results generated by \ac{eeg} signal classes between these two populace. There is also the lack of definition of a specific threshold in case of \ac{erp}s and neural synchrony. The existence of this overlap and lack of a threshold in separating \ac{sz} populations and other populations has not allowed the development of a clinically acceptable \ac{eeg} based point of care prognostic tool for \ac{sz}.

Most research works employing \ac{eeg} signals focus on using one class of \ac{eeg} signal such as the \ac{mmn}, \ac{assr}, etc. in discriminating between \ac{sz} populace and the others. And some others investigate the  functional dysconnectivity model by making use of coherence and synchrony measures to describe spatial disconnectivity of brain regions for defined activities to be carried out. 

The \ac{mmn} has over time been the most stable marker for \ac{sz} but has a problem of overlap between the \ac{sz} populace and non-\ac{sz} populace, also it has no definite threshold of attenuation levels that can be used to discriminate between \ac{sz} populations and other populations. Thus there is the need to investigate methods of improving the results of \ac{mmn} by reducing or eliminating the overlap between \ac{sz} and other populations in the diagnostic results generated by \ac{mmn}.


\section{Problem Statement}\label{sec:Problem Statement}
\ac{sz} is a mental ailment mostly identified during psychosis through psychiatric nosology and one whose symptoms manifest as a parent class of other mental ailments. Early(pre-psychotic) detection and treatment of \ac{sz} has proved that transition into psychosis can be prevented. As diagnosis of \ac{sz} currently depends on symptoms of the psychotic phase, early treatment cannot be administered. Mental health diagnostics depend on the subjective evaluations of trained clinicians (albeit based on the considerable experience and judgement).There exists no objective scientific instrument to diagnose mental disorders, thus the need for psychosis. This means that in cases of schizophtrenia, the patient has to be exrtremely sick, before they can be identified as sick. Prevention is better than cure, as the financial, social and emotional expense of curative treatments are mostly expensive. Since \ac{sz} can be managed well in the pre-spychotic stage and such management usally prevents conversion to psychosis. This project aims to develop an instrument for early detection and diagnosis of \ac{sz}. 
Similar works have been done in the past but seem to have hit a standstill as there is a consistent exisint goverlap between the \ac{sz} and non \ac{sz} population in the results provided by these methods. These methods usually utilize one \ac{eeg} signal class indicative of \ac{sz} state and mostly utilize \ac{mmn} from \ac{eeg} signals. This study will be conducted to improve the accuracy and specificity of \ac{mmn} as a marker for \ac{sz} by utilising a novel method of combining \ac{mmn} computed features with other \ac{fep} modalities including \ac{assr} and measures of connectivity and complexity such as fuzzy entropy. Then this project will evaluate the possibility of developing an objective, accurate and easy to administer test for \ac{sz} based on the study results.

\section{Aims and Objectives}\label{sec:aims_objective}
The aim of this project is to develop an instrument for improved diagnosis and management of \ac{sz}. Hopefully in the future, a \ac{poc} device for SZ based on this work will be created.
Specific objectives of the project are to:
\begin{enumerate}
	\item Consuct literature review on possible EEG features that can be used as markers for SZ diagnois.
	\item Develop a signal processing pipeline for EEG signal classification towards SZ diagnosis.
	\item Acquire EEG data from \ac{sz} patients and normal subjects.
	\item Evaluate the accuracy of the system in diagnosing \ac{sz}.
\end{enumerate}

\section{Scope of Project}\label{sec:projectScope}
This study will focus on employing statistical, analytical and computer learning methods in understanding the distribution of \ac{mmn} results across schizophrenic and non-schizophrenic populations.
This study will also investigate the best feature extraction, combination and computer learning methods to be employed in developing \ac{mmn} and \ac{sz} classifiers. 
This study will correlate the results of the classifier with the type and severity of the \ac{sz} cases.
This study will at no point seek to understand the neuronal, genetic or environmental workings that contribute to \ac{sz}.
At no point will this study question the existing psychiatric evaluation methods or try to understand them. Being blind to the basis of psychiatric evaluation and nosology is important so as not to introduce any form of bias during data processing.

\section{Outline of Thesis}\label{sec:reportOutline}
While chapter two discusses the methods adopted by previous works with similar aim to this study and their results, 
chapter three will later on explain the methodolgy to be adopted for this work, explaining the reasons for the adopted 
methodology. Chapter four will at the end of this project present the results obtained and more on their interpretation in the discussion sub-section. 
Chapter five will at the end of this project draws conclusions from the results and suggests methods further works should use based on 
the results obtained and hallmarks achieved in this study.



