
\chapter{Literature Review}\label{Ch:2}
\vspace{30pt}


\section{Neuropathology and Brain anomalies of Schizophrenia}\label{sec:Neuropathology_Brain_Anomalies}
%%%%subsection
\subsection{Neuropathological Findings Overtime}\label{Neuropathology}
Most neuropathological findings in schizophrenia show that the identified neuropathological causes do not qualify for schizophrenia markers, due to their unique and dynamic evolution in patients and some due to incosistencies across the schizophrenia population.
\begin{quotation}
	\textit{The neuropathology of schizophrenia remains obscure despite the fact that many neuropathologists have investigated this area for over 100 years. While remarkable progress has been made in the neuropathological study of neurodegenerative diseases including Alzheimer's disease, progress in studying the neuropathological entity of schizophrenia has not kept pace; the phrase “schizophrenia is the graveyard of neuropathologists” has been stated in the field}
	\begin{flushright}
		\cite{iritani2007neuropathology}
	\end{flushright}
\end{quotation}
\begin{quotation}
	\textit{At the neuronal level, the hypothesis of reduced neuropil as a neuropathology of schizophrenia has been mostly disproved as they have been shown to be associated more with age interactions. Rather a quantitative summary suggests that neuron density is increased in patients compared to controls.}
	\begin{flushright}
		\cite{bakhshi2015neuropathology}
	\end{flushright}
\end{quotation}
The hypothesis of reduced neuropil(a dense network of interwoven nerve fibres, their branches and synapses, together with glial filaments) is onw among many neuropathological findings in \ac{sz} that has been disproved, however, other findings have proved to be incosistent across \ac{sz} populations.
\begin{quotation}
	\textit{There is evidence of decreased inhibitory neurons in schizophrenia. This provides support for the involvement of inhibitory cortical circuits in the development and maintenance of schizophrenia.}
	\begin{flushright}
		\cite{bakhshi2015neuropathology}
	\end{flushright}
\end{quotation}
\begin{quotation}
	\textit{Post-mortem neuropathologial abnormalities have been found in almost all areas of the brain, but there are more reports describing the temporal lobe(auditory) and frontal lobe(language, expression,cognition) compared to those describing other areas of the brain.}
	\begin{flushright}
		\cite{iritani2007neuropathology}
	\end{flushright}
\end{quotation}
Thus the prominence of neuropathological anomalies in the temporal and frontal lobe could explain why auditory hallucinations and cognitive deficienceis is prominent in \ac{sz} patients, but these anomalies vary across patients.
\begin{quotation}
	\textit{Computer imaging studies using statistical analysis and immunohistological techniques has led to post-mortem neuropathological examination of the brains of schizophrenia patients, over the last 20 years it is clear that schizophrenia is not a pure functional disease without organic factors.}
	\begin{flushright}
		\cite{iritani2007neuropathology}
	\end{flushright}
\end{quotation}
\begin{quotation}
	\textit{Most neuropathological abnormalities are rationally explained by the hypothesis of a neurodevelopmental disorder in schizophrenia.}
	\begin{flushright}
		\cite{iritani2007neuropathology}
	\end{flushright}
\end{quotation}
\begin{quotation}
	\textit{Studies over time have shown that the neuropathological picture of schizophrenia is not static but changes over time, and indicates that age and length of illness are relevant variables in any analysis..}
	\begin{flushright}
		\cite{bakhshi2015neuropathology}
	\end{flushright}
\end{quotation}
\ac{sz} being views as a neurodevelopmental disorder suggests changes in its root mechanisms overtime, but consistent behavioural readouts in its patients. This shows that \ac{sz} might need to be studied at higher levels of organization ti develop diagnostic tools for the ailment. The eveolution of \ac{sz} pathology thus suggests that:
\begin{quotation}
	\textit{A single, fixed pathological description may no longer be expected to reflect the complex nature of changes in development, adult plasticity and ageing in schizophrenia studies.}
	\begin{flushright}
		\cite{bakhshi2015neuropathology}
	\end{flushright}
\end{quotation}

%%%%%subsection
\subsection{Consistent Brain Anomalies Overtime}\label{brain_anomalies}
Certain anomalies of the brain are however consistent with the schizophrenic population and have withstood the test of time. Overtime the following findings have remained consistent:
\begin{quotation}
	\textit{Imaging studies reveal certain differences in schizophrenia including larger ventricles, decreased grey matter, smaller hippocampus, decreased asymmetry and altered gyrification, in schizophrenia pathology.}
	\begin{flushright}
		\cite{bakhshi2015neuropathology}
	\end{flushright}
\end{quotation}
Certain regional and functional anomalies have been consistent in the brain over years of research despite the evolving nature of \ac{sz}, one of such is:
\begin{quotation}
	\textit{The frontal and temporal lobes and their thalamic relays appear particularly affected in schizophrenia and there exists evidence for significant asymmetries. These regions are consistently affected in patients having neurodegenerative cognitive impairment.}
	\begin{flushright}
		\cite{halliday2001proceedings}
	\end{flushright}
\end{quotation}
\begin{quotation}
	\textit{Differences in volumes of various forebrain subregions have been reported, but with less consistent results, the lack of consistency being a hallmark of schizophrenia pathology.}
	\begin{flushright}
		\cite{bakhshi2015neuropathology}
	\end{flushright}
\end{quotation}
So therefore,
\begin{quotation}
	\textit{Schizophrenia may involve a spectrum of brain network changes that are only unified in the extent to which there are all deviations from the healthy equilibrium.}
	\begin{flushright}
		\cite{bakhshi2015neuropathology}
	\end{flushright}
\end{quotation}
This suggests that empirical diagnosis of schizophrenia may only take place at higher levels of organiztion in the brain. So there is the question of how genetics contribute to schizophrenia and how does schizophrenia affects the brain structure at the functional level of organization?

\section{The Role of Genetics in Schizophrenia}\label{sec:geneticsSZ}
\begin{quotation}
	\textit{There exists multi-factorial genetic heritability in schizophrenia, for which studies have provided evidence of diversity of involved phenotype in patients.}
	\begin{flushright}
		\cite{bakhshi2015neuropathology}
	\end{flushright}
\end{quotation}
\begin{quotation}
	\textit{Recent molecular biology studies have reported some putative genes, some of which may have the function of neurodevelopment or making neuronal networks.}
	\begin{flushright}
		\cite{iritani2007neuropathology}
	\end{flushright}
\end{quotation}
\begin{quotation}
	\textit{Structurally compromised cellular phenotypes have yet to be definitively identified in schizophrenia and the relationship between any neurochemical and structural abnormalities remain unclear. This is largely due to the subtle changes that take place in these components and their diffuse nature.}
	\begin{flushright}
		\cite{halliday2001proceedings}
	\end{flushright}
\end{quotation}
\begin{quotation}
	\textit{The several identified gene loci in schizophrenia leave many questions unanswered including the therapeutic significance of these gene loci in schizophrenia.}
	\begin{flushright}
		\cite{harrison2015recent}
	\end{flushright}
\end{quotation}
\begin{quotation}
	\textit{There exist more unidentified gene loci which may result from rare variants, gene-gene and gene-environment interactions. The biological significance of these gene loci remains unclear and may possess more therapeutic information.}
	\begin{flushright}
		\cite{harrison2015recent}
	\end{flushright}
\end{quotation}
\begin{quotation}
	\textit{The identification of genes and loci, though a triumph, is merely the start of a long process towards meaningful biological understanding, let alone better treatment of the disorder.}
	\begin{flushright}
		\cite{harrison2015recent}
	\end{flushright}
\end{quotation}
\begin{quotation}
	\textit{Genetic studies have consistently suggested a genetic overlap of schizophrenia with bipolar disorder and neurodevelopmental disorders such as autism.}
	\begin{flushright}
		\cite{van2010genome}
	\end{flushright}
\end{quotation}
\begin{quotation}
	\textit{Genomic wide association studies are identifying novel common and rare genetic variants associated with psychotic disorder such as schizophrenia.}
	\begin{flushright}
		\cite{van2010genome}
	\end{flushright}
\end{quotation}
\begin{quotation}
	\textit{Associations between common and uncommon genetic variants and schizophrenia, though statistical facts, are not necessarily indexes of causal pathways.}
	\begin{flushright}
		\cite{henriksen2017genetics}
	\end{flushright}
\end{quotation}
\begin{quotation}
	\textit{Many of the discovered genetic associations in schizophrenia cases are in fact non-specific to schizophrenia but indicative of a genetic vulnerability to several mental disorders.}
	\begin{flushright}
		\cite{henriksen2017genetics}
	\end{flushright}
\end{quotation}
\begin{quotation}
	\textit{Genetic discoveries in schizophrenia come with the sobering realisation that the genetic basis of schizophrenia is more complex in many ways than had generally been anticipated.}
	\begin{flushright}
		\cite{harrison2015recent}
	\end{flushright}
\end{quotation}
Therre has always been and will always be genetic discoveries in relation to \ac{sz}, but the variations in discovery across populations and the amoun of understanding of these discoveries limits the use of genes in developing diagnostics tools for \ac{sz}.

\section{Cortical Organization in Schizophrenia}\label{sec:corticals}
Genetics and neuropathology might be complex, but engineers and the natural existence of abstractions and pyramidal flow in structures is always to the rescue.
\begin{quotation}
	\textit{There exists structural  brain abnormalities of developmental origin and neuropsychological deficits in schizophrenia.}
	\begin{flushright}
		\cite{raggi2022auditory}
	\end{flushright}
\end{quotation}
\begin{quotation}
	\textit{There is increasing evidence that altered cortical oscillatory activity may be associated with neuropsychiatric disorders such as schizophrenia, that involve dysfunctional cognition and behavior.}
	\begin{flushright}
		\cite{uhlhaas2008role}
	\end{flushright}
\end{quotation}
\begin{quotation}
	\textit{Schizophrenia is associated with abnormal cortical oscillatory activity in a wide range of frequencies with disturbed large-scale synchronization of oscillations. These are consistently associated with core cognitive dysfunctions and symptoms of disorder suggesting a causal relation.}
	\begin{flushright}
		\cite{uhlhaas2008role}
	\end{flushright}
\end{quotation}
\begin{quotation}
	\textit{The neurodevelopmental hypothesis of schizophrenia is consistent with the involvement and possible dysfunction of neural oscillations in early development of cortical circuits and with delayed manifestation of the disorder in late adolescence.}
	\begin{flushright}
		\cite{uhlhaas2008role}
	\end{flushright}
\end{quotation}
Among known cortical behaviours, some are more prominent among certain populations of \ac{sz} patients, some of which are the \ac{chr}, \ac{fep} and among non \ac{fep} such as the \ac{hc}.
\begin{quotation}
	\textit{There is consistently widespread smaller cortical volume among \ac{chr} schizophrenia patients compared with healthy control subjects, particularly among the younger group.}
	\begin{flushright}
		\cite{chung2019cortical}
	\end{flushright}
\end{quotation}
\begin{quotation}
	\textit{There exists similar cortical changes in CHR who later go into psychosis or remit from schizophrenia.}
	\begin{flushright}
		\cite{chung2019cortical}
	\end{flushright}
\end{quotation}
Some altered function in \ac{sz} patients are reflected prominently in certain corticla regions. We discuss some of those consistent across previous works here.
\begin{quotation}
	\textit{Poor premorbid functioning in childhood is associated with smaller surface area in orbitofrontal, lateral frontal, rostral anterior cingulate, precuneus and temporal regions.}
	\begin{flushright}
		\cite{chung2019cortical}
	\end{flushright}
\end{quotation}
\begin{quotation}
	\textit{There exists an association between decreased dorsolateral prefrontal cortex activity and connectivity and a task-related neural network in schizophrenia patients.}
	\begin{flushright}
		\cite{yoon2008association}
	\end{flushright}
\end{quotation}
\begin{quotation}
	\textit{There may exist imbalance across multiple regions of the neocortex in imaging studies of schizophrenia patients. }
	\begin{flushright}
		\cite{rolls2021attractor}
	\end{flushright}
\end{quotation}
\begin{quotation}
	\textit{Short term treatments are associated with prefrontal cortical thinning in schizophrenia patients and treatment is associated with better cognitive control and increased prefrontal functional activity.}
	\begin{flushright}
		\cite{lesh2015multimodal}
	\end{flushright}
\end{quotation}
\begin{quotation}
	\textit{The whole brain volume tends to correlate with the measures of general intelligence as well as with a range of specific cognitive functions in control and female schizophrenia patients.}
	\begin{flushright}
		\cite{antonova2004relationship}
	\end{flushright}
\end{quotation}
\begin{quotation}
	\textit{Inhibitory signalling in the prefrontal cortex cortex is required for gamma oscillatory activity and working memory function; and disturbances in this signalling contribute to altered gamma oscillations and working memory changes in schizophrenia.}
	\begin{flushright}
		\cite{rolls2021attractor}
	\end{flushright}
\end{quotation}
\begin{quotation}
	\textit{Archicortical but not paleocortical, prefrontal cortex tends to associate with the measures of executive function.}
	\begin{flushright}
		\cite{antonova2004relationship}
	\end{flushright}
\end{quotation}
\begin{quotation}
	\textit{Temporal lobe, hippocampus, and parahippocampal gyrus correlates with cognitive abilities such as performance speed and accuracy, memory and executive function, verbal endowment and abstraction/categorization, some of which are specific to schizophrenia.}
	\begin{flushright}
		\cite{antonova2004relationship}
	\end{flushright}
\end{quotation}
The consistent behaviour of cortical regions with brain functions significantly impaired in schizophrenia indicate the ability of cortical level brain data in developing diagnostic and prognostic tools for schizophrenia.

\section{Electroencephalogram and Schizophrenia}\label{sec:EEG}
Various brain electroencephalogram signals are indicative of cortical activity and significant change in brain state in response to stimuli. This signal classes can detect deviations of cortical activities from the normal. The correlation of cortical activty with degrading cognitive function in schizophrenic populations can be monitored using certain electroencephalogram signal class and can be used to understand the prognosis of schizophrenia and get insight into diagnostics of the condition. We discuss some of this signals below and state finding s consistent from review of past works found in \ac{mmn} and \ac{assr}.
\begin{quotation}
	\textit{\ac{n100} EEG component which is a negative peak for auditory stimuli in response to infrequent deviant auditory stimuli and 150ms for visual stimuli is consistently reduced among schizophrenia patients and first degree relatives. It is also associated with severity of symptoms in healthy, clinical high risk patients and psychotic children. There occurs deficiency in N100 suppression during self generated speech in schizophrenia patients and first degree relatives, alsoduring vocalization in clinical high risk patients relative to healthy control subjects.}
	\begin{flushright}
		\cite{hamilton2020electroencephalography}
	\end{flushright}
\end{quotation}
\begin{quotation}
	\textit{The P50 EEG component is elicited during sensory gatind paradigm in response to pairs of auditory stimuli separated by 500ms interstimulus interval,larger at first stimulus(S1) and suppressed at second stimulus(S2) which reflects gating out of irrelevant information. Deficiency in gating in terms of ratio of S2 to S1 has been consistently shown in schizophrenia patients.}
	\begin{flushright}
		\cite{hamilton2020electroencephalography}
	\end{flushright}
\end{quotation}
\begin{quotation}
	\textit{Repetition positivity is an EEG component elicited by standards that increase with successive standard repetitions, they are consistent with strengthening of standard memory trace and are associated with prediction that such standard will reoccur. There exist deficiency in clinical high risk patients of schizophrenia for earliest appearing standards and more prominently for late appearing standards within local sequences of repeating standards following each deviant.}
	\begin{flushright}
		\cite{hamilton2020electroencephalography}
	\end{flushright}
\end{quotation}
\begin{quotation}
	\textit{\ac{eeg} \ac{p300} component is elicited in the process of choice/decisions making on stimuli requiring response(P3a subcomponent) and those requiring no response(P3b subcomponent). P3a and P3b have consistently been reduced in schizophrenia patients and first degree relatives. P300 amplitude reduction may reflect genetic risk for schizophrenia. The P300 in some research works have predicted future psychosis onset time, differentiated converters and non-converters, predicted clinically high risk patients remission, predicted improvement in negative and general psychopathology symptoms and has been shown to be linked to some neuronal activity and hormones. The P300 has been localised to the temporo-parietal junction with amplitude deficiency potentially implicating compromise of these regions in those at greater risk.}
	\begin{flushright}
		\cite{hamilton2020electroencephalography}
	\end{flushright}
\end{quotation}
\begin{quotation}
	\textit{The mismatch negativity occurs in response to violations of a rule established by a sequence of sensory stimuli.}
	\begin{flushright}
		\cite{etiologySZ}
	\end{flushright}
\end{quotation}
\begin{quotation}
	\textit{MMN generators have been localised to the auditory cortex and frontal cortex and its generation  has been mapped to the theta frequency band in man.}
	\begin{flushright}
		\cite{alho1995cerebral}
	\end{flushright}
\end{quotation}
\begin{quotation}
	\textit{MMN can distinguish future remitters from non-remitters and predict later functional recovery and has been shown to predict shorter time to conversion among clinical high risk patients of schizophrenia.}
	\begin{flushright}
		\cite{hamilton2020electroencephalography}
	\end{flushright}
\end{quotation}
\begin{quotation}
	\textit{There exists resting state EEG abnormalities in schizophrenia patients including increased delta, theta power and reduced alpha power.}
	\begin{flushright}
		\cite{boutros2008status}
	\end{flushright}
\end{quotation}
\begin{quotation}
	\textit{EEG spectral abnormalities which predict psychosis conversion including increased delta, theta power alone or in combination with symptom severity and decreased alpha peak frequency have been identified.}
	\begin{flushright}
		\cite{gschwandtner2009eeg}
	\end{flushright}
\end{quotation}
\begin{quotation}
	\textit{Schizophrenia is associated with gamma band abnormalities(30-80)Hz implicated in sensory registration, cross-modal sensory integration and higher order cognitive functions.}
	\begin{flushright}
		\cite{uhlhaas2010abnormal}
	\end{flushright}
\end{quotation}
\begin{quotation}
	\textit{Gamma auditory steady state response(ASSR) are the most replicated gamma oscillation abnormalities in schizophrenia.}
	\begin{flushright}
		\cite{hamilton2020electroencephalography}
	\end{flushright}
\end{quotation}
\begin{quotation}
	\textit{There is evidence for reduced alpha event-related desynchronization in schizophrenia patients and clinical high risk patients relative to healthy control subjects.}
	\begin{flushright}
		\cite{koh2011meg}
	\end{flushright}
\end{quotation}

\subsection{Mismatch Negativity in Schizophrenia}
\begin{quotation}
	\textit{Mismatch Negativity is a negative component of the event-related response in an EEG signal, elicited by any perceptible change in some repetitive aspect of an auditory stimulation (e.g., stimulus pitch, stimulus duration).}
	\begin{flushright}
		\cite{goodwin2007heartland}
	\end{flushright}
\end{quotation}
\begin{quotation}
	\textit{The mismatch negativity (MMN) is a brain response to violations of a rule, established by a sequence of sensory stimuli (typically in the auditory domain)}
	\begin{flushright}
		\cite{etiologySZ}
	\end{flushright}
\end{quotation}
\begin{quotation}
	\textit{MMN is elicited following a deviation in any sound characteristic, amplitude, frequency, intensity, duration, location.}
	\begin{flushright}
		\cite{etiologySZ}
	\end{flushright}
\end{quotation}
\begin{quotation}
	\textit{There exists some ERP components whose characteristics are consistent with schizophrenia that exist simultaneously with MMN.}
	\begin{flushright}
		\cite{de2020pearls}
	\end{flushright}
\end{quotation}
\begin{quotation}
	\textit{The P3a component occurs 250-280ms post-stimulus alongside MMN.}
	\begin{flushright}
		\cite{comerchero1999p3a}
	\end{flushright}
\end{quotation}
\begin{quotation}
	\textit{The P3a is associated with activity related to engagements of attention, shift in attention and novelty processing.}
	\begin{flushright}
		\cite{friedman2001novelty}\cite{polich2007updating}
	\end{flushright}
\end{quotation}
\begin{quotation}
	\textit{P3a has been found to be consistently impaired in schizophrenia and like MMN is a strongly automatic process.}
	\begin{flushright}
		\cite{grillon1990increased}\cite{jahshan2012automatic}\cite{koshiyama2022neuroimaging}
	\end{flushright}
\end{quotation}
\begin{quotation}
	\textit{Impaired early auditory processing is a well characterised finding in schizophrenia that is theorised to contribute to clinical symptoms.}
	\begin{flushright}
		\cite{kim2020neurophysiological}
	\end{flushright}
\end{quotation}
\begin{quotation}
	\textit{Auditory mismatch negativity is consistently impaired in schizophrenia as well as in bipolar disorder with psychotic features, though to a lesser extent in bipolar disorder.}
	\begin{flushright}
		\cite{raggi2022auditory}
	\end{flushright}
\end{quotation}
\begin{quotation}
	\textit{Auditory mismatch negativity shows the involvement of the N-methyl-D-aspartate receptor in the pathophysiology of both bipolar disorder and schizophrenia.}
	\begin{flushright}
		\cite{raggi2022auditory}
	\end{flushright}
\end{quotation}
\begin{quotation}
	\textit{Auditory mismatch negativity may be considered as a correlate of a common psychopathology of schizophrenia and bipolar spectrum illnesses.}
	\begin{flushright}
		\cite{raggi2022auditory}
	\end{flushright}
\end{quotation}
\begin{quotation}
	\textit{Duration deviant MMN might be more sensitive to schizophrenia than frequency deviant MMN among first episode patients.}
	\begin{flushright}
		\cite{umbricht2005mismatch}
	\end{flushright}
\end{quotation}
\begin{quotation}
	\textit{MMN can be sourced for near the primary auditory cortex, in the hemisphere contralateral to the ear of stimulation and the frontal region mainly involving the right hemisphere.}
	\begin{flushright}
		\cite{raggi2022auditory}
	\end{flushright}
\end{quotation}
\begin{quotation}
	\textit{MMN is quantified by subtracting the evoked response to the standard tone from the corresponding response to the deviant stimulus.}
	\begin{flushright}
		\cite{ferri2003mismatch}
	\end{flushright}
\end{quotation}
\begin{quotation}
	\textit{MMN is more evident on the frontal sites and on the mastoids due to dipole inversion.}
	\begin{flushright}
		\cite{alho1986separability}
	\end{flushright}
\end{quotation}
\begin{quotation}
	\textit{MMN is considered as a highly reproducible neurophysiological marker.}
	\begin{flushright}
		\cite{javitt2000intracortical}\cite{umbricht2005mismatch}
	\end{flushright}
\end{quotation}
\begin{quotation}
	\textit{MMN has been shown to follow a progressive course, with reduced MMN amplitude associated with a loss of grey matter in the left superior temporal gyrus.}
	\begin{flushright}
		\cite{salisbury2007progressive}
	\end{flushright}
\end{quotation}
\begin{quotation}
	\textit{There exists significant associations between MMN and psychotic symptoms.}
	\begin{flushright}
		\cite{hall2007genetic}
	\end{flushright}
\end{quotation}
Most works on mismatch negativity show that they are consistent biomarkers in characterising schizophrenia.

\subsection{Auditory Steady State Response in Schizophrenia}
\begin{quotation}
	\textit{Most of the studies that show abnormalities of gamma oscillations in the scalp-recorded EEG of schizophrenics and thus hypothesise their reflection neural circuit abnormalities has come from studies of ASSRs in which simple auditory stimulus such as clicks are delivered at rapid rates and entrain the EEG at the stimulation frequency.}
	\begin{flushright}
		\cite{spencer2008gamma}
	\end{flushright}
\end{quotation}
\begin{quotation}
	\textit{Chronic schizophrenics show a gamma ASSR deficit with reduced power and phase-locking of ASSRs in the gamma band but not at lower frequencies compared with healthy individuals and seems to be most pronounced for 40Hz stimulation.}
	\begin{flushright}
		\cite{hong2004evoked}
	\end{flushright}
\end{quotation}
\begin{quotation}
	\textit{ASSR deficit has also been reported in early onset psychosis suggesting that it might be a manifestation of a neural circuit disorder (or set of disorders) that is shared by psychosis in general.}
	\begin{flushright}
		\cite{wilson2008cortical}
	\end{flushright}
\end{quotation}
\begin{quotation}
	\textit{ASSR deficit has been reported consistently as being consistent with grey matter volume loss, which is a consistent change in brain structure of schizophrenics.}
	\begin{flushright}
		\cite{salisbury2007progressive}
	\end{flushright}
\end{quotation}
\begin{quotation}
	\textit{Some of the neural circuitry abnormalities underlying the gamma ASSR deficit might be common to psychosis in general, whereas others might be specific to particular disorders.}
	\begin{flushright}
		\cite{spencer2008gamma}
	\end{flushright}
\end{quotation}
\begin{quotation}
	\textit{ASSR evoked power at 40Hz is reduced in schizophrenics compared to healthy control subjects.}
	\begin{flushright}
		\cite{spencer2008gamma}
	\end{flushright}
\end{quotation}
\begin{quotation}
	\textit{20Hz ASSR does not differ between groups, but phase-locking and evoked power of the 40Hz harmonic of the 20Hz ASSR are reduced in both SZ and affective disorder patients. Phase-locking of this 40Hz harmonic was correlated with total positive symptoms in SZ.}
	\begin{flushright}
		\cite{spencer2008gamma}
	\end{flushright}
\end{quotation}
\begin{quotation}
	\textit{The gamma ASSR deficit is present at first hospitalisation for both schizophrenia and affective disorder but shows a left hemisphere bias in first hospitalised SZ.}
	\begin{flushright}
		\cite{spencer2008gamma}
	\end{flushright}
\end{quotation}