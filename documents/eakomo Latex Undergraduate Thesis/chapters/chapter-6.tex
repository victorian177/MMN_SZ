\chapter{Conclusion and Future Works}\label{Ch:6}
Wireless communication  networks designers and operators are faced with the ever-increasing demands for wireless radio network resources and improved QoS provisioning. These challenges are driven by the always best connected (ABC)  access network  demands through network selection by wireless network users,  constant connectivity at anywhere demands by users and the explosive growth of multimedia-hungry devices, such as smartphones, tablets,  etc. within the wireless communication networks. Efficient radio resource utilization and QoS provisioning are major issues for wireless communication networks. \par 

 In this thesis, the use of MCDM algorithm, call admission control scheme, and cooperative communication techniques to improve the ABC network experience, enhance the radio resource utilization, and QoS provisioning for wireless network users are explored. Section \ref{sec:SummaryofContributions6}  gives the summaries of thesis's contributions. Finally, Section \ref{sec:FutureWork6}  highlights some important directions for future work.
 
\section{Summary of Contributions} \label{sec:SummaryofContributions6}
 In Chapter \ref{Ch:1}, an overview of current  challenges in radio resource management  of wireless communication networks is provided. The chapter introduces the research problems addressed in this research work, the motivations, the research questions, and  research objectives. In addition, the research contributions and scope, as well as the outline on the thesis organization are also provided. In Chapter \ref{Ch:2}, the background topics for the thesis the is presented. \par
 
Chapter \ref{Ch:3}  studied the problem of enhancing QoS for wireless  MoH  based cellular networks. An  ATMA  CAC algorithm is proposed. The ATMA CAC  exploits the knowledge of the  calls generation pattern and  moving phase and stationary phase events of the MoH networks to enhance the new call blocking probability, handoff call dropping probability, and channel resource block utilization of the moving wireless networks. The  ATMA scheme adaptively smoothens out the new calls, depending on the mobility events of the HSRC. A  framework to investigate the effect of varying  load traffic on the new call blocking probability, handoff call dropping probability, and resource block utilization of the proposed scheme  using Markov chain analysis is developed.  The simulation results demonstrated that the ATMA  CAC scheme  improved  resource block utilization; with reduced new call and handoff call dropping probabilities compared to NMAT  CAC scheme.  \par 
 
Chapter \ref{Ch:4}  investigated the problem of throughput and QoS enhancement in wireless cellular networks with cell-edge users. A BCC scheme that  opportunistically  engage UES in cooperative communication and  exploits the transmission channel qualities variation with adaptive modulation coding to enhance the average throughput, QoS, and radio resource utilization in the cellular wireless  networks is proposed.  An analytical framework, using Markov chain process to investigate the effect of the queue buffer size on the average throughput, call  blocking probability, and radio resource utilization under varying inter-arrival rates (network  load),  UE cooperative relay  and cell-center zone  indice of the wireless networks is developed. A platform to investigate the benefit/gain performances of the  BCC scheme for improving the network bit-rate transmission, radio resource utilization, and blocking probability of  UEs' compared to NBCC scheme is provided. The simulation results showed that the proposed BCC scheme achieved a significant improvement in the average  throughput,  data blocking probability, and radio resource utilization for wireless communication networks compared to the NBCC scheme. \par 
 
In Chapter \ref{Ch:5},  the problem of access network selection  for single and group calls in HWNs is investigated.  The application of a new MCDM algorithm named  MULTIpliative Multi-Objective Optimization Ratio  Analysis (MULTIMOORA) for VHO in HWNs and compared it with some traditionally popular  MCDM schemes: SAW, GRA, VIKOR, and TOPSIS.   A detailed performance evaluation of the proposed scheme  was conducted for single calls.  The results demonstrate that MULTIMOORA outperformed SAW, GRA, and VIKOR in terms of access network selection for single calls (voice, file-download, and video-streaming), while having similar access network selection performance with TOPSIS for the  access  network's single call  traffic considered. 



The access network selection for group calls in HWNs with dynamic criteria investigations  demonstrated that TOPSIS algorithm utilizes the  benefits of  WLAN more than MULTIMOORA algorithm  at low speed, while MULTIMOORA algorithm  utilizes the benefits of LTE more than TOPSIS algorithm  at above low speed region for both high priority file-download and video-streaming group calls.   The  performance of TOPSIS is observed to be relatively more unstable at the high speed\textendash region, unlike MULTIMOORA's.
 The relative stability of MULTIMOORA comes from the reinforcement and integration of the  three decision making techniques of the MULTIMOORA. Furthermore from these findings, it can be suggested that  an  hybrid algorithm between MULTIMOORA and TOPSIS could produce an improved  access network selection algorithm for NGWNs.


 
\section{Future Work}\label{sec:FutureWork6}
 
 \subsection{Handover management in MoH networks} \label{subsec:MobileHotspot}
The NGWNs, such as 5G networks are proposed to be ultra-dense heterogeneous wireless communication networks.  Handover management in high mobility wireless network systems is still a major challenge that calls for further investigations. The seamless and efficient handover between different MeNBs and RATs remain an open area for future research.

 \subsection{ MoH networks with traffic load predictions} \label{subsec:loadPredictionMobileHotspot}

The wireless environment of practical MoH  networks is very dynamic. Some statistical characteristics of the MoH networks,
such as new arrival rate, handoff call arrival rates, and call lifetime  are time-varying. In MoH wireless network environment where  
these traffic load parameter values are not well known, the CAC algorithm  of  such MoH networks could be plagued  with inefficient management of the radio resources. An interesting solution to this problem would be to exploit network traffic load  predicting techniques.  Predicting techniques,  such as  Least Squares algorithm, Machine learning algorithm, Grey predictor scheme, would allow the mobile hotsot network to obtain the statistical parameters (call arrival rate and call life time) without prior knowledge. Hence, the investigation into the performance of MoH networks  with and without traffic load prediction scheme would  be an interesting extension for future work.



 \subsection{ Security of BCC }\label{subsec:Securityofbufferedcooperativecommunication}
In the BCC scheme,  it is possible for malicious relaying UE nodes to attack the network. Some UEs can possibly behave in a malicious manner by intentionally trying to corrupt the communication by sending  garbled signal to the  destination.  This action of  malicious relaying UEs can degrade the performance of the network.  Considering the effect of malicious  of relay  UE nodes attack and how to mitigate their actions  for ensuring the reliability the networks will be of  significant future research importance. 





 \subsection{ Inter-cell-interference in cellular  network communication with BCC deployment}
 
 In cellular wireless communication  networks, Inter-cell-Interference (ICI) is a predominant problem, especially among cell-edge UEs.  ICI problem can significantly degrade the QoS experience of the cell-edge UEs. The effect of the problem of ICI on the QoS experience for edge-cell UEs  in wireless cellular networks could  be worsen, with the deployment of 
  BCC network scheme, where  
 cell-center UEs  are engaged  opportunistically as relay nodes. Therefore, as future work  there be would  the need to investigate/examine the impact of BCC on ICI management in cellular communication networks, by comparing  the ICI performance of  cell-edge networks with BCC deployment and with NBCC deployment.
 
 
 
 
 
 \subsection{Access network selection}\label{subsec:Networkaccessselection}
The number of available access network alternatives within  HWNs can sometimes  fluctuate. One or more RATs can breakdown, and  its or their services become unavailable within the HWNS. Sometime more RATs can become activated, thereby increasing the number of available access networks within the HWNs. These variations have  been reported to affect the access network selection decisions of  MCDM  algorithms leading to   rank reversal/abnormality.  Further work should investigate  the rank reversal/abnormality and  sensitivity analysis  of performance of  MULTIMOORA compared to  TOPSIS.  Also, the interaction of criteria influence using ANP needs to be  investigated.
 
  
