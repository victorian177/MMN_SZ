\chapter{Introduction}\label{Ch:1}		% TODO: you can change the chapter title from Introduction if necessary

\section{Background}\label{sec:rehabintroduction}	
There has always been a constant thirst to improve the quality of human life. The effects of disabling disorder can wreak havoc on a person’s life and their daily livelihood. Therefore there is need for more rehabilitation methods readily available for patients to be able to make the most of their early recovery period to be able to gain their ability back \cite{Langhorne2011a}. This study is targeted at building a bilateral rehabilitation robot that can help in Nigeria particularly and for Africa.

\section{Aims and Objectives}\label{Aims}
This research is aimed at designing a proof of concept of a Bilateral Rehabilitation robot, with a low-cost multi-axis force torque sensor to go along with it. Some of the Objectives of this project are:
\begin{itemize}
	\item To Design and Develop the Bilateral Rehabilitation Robot \ac{blue}
	\item To design a multi-axis force sensor out of 1D load cells
	\item To test and compare the novel force sensor with a commercial multi-axis force sensor
	\item to test the effectiveness of the bilateral rehabilitation robot over a unilateral one
	\item to apply Transfer Learning to improve the performance of the load cell assembly %new
\end{itemize}

\section{Project Justification}\label{Justification}
This Project is brought about by the need for more rehabilitation means for patients, who have suffered from a Stroke. For example in Nigeria, there is a very large lack of personnel to patients which means most affected patient will never be able to get the care needed to make them feel and become better after a stroke.

\section{Scope of the Project}\label{scope}
This Project, even though it is about eventually building a full Bilateral rehabilitation robot and a new multi-axis force sensor, the scope has been reduced to allow for completion within the given timeframe.\\
The deliverables that must be completed are:
\begin{itemize}
	\item Construction of a proof-of-concept Bilateral rehabilitation robot.
	\item Development of a multi-axis force sensor.
	\item Performing comparison of the novel force sensor and the commercial force sensor.
\end{itemize}

