
\chapter{Additional Materials}\label{ch:litrev}
%\appendix
\section{Codes Used for this project}
\subsection{arduino code for reading forces from 3 load cells}


%arduino code c
\lstinputlisting[language=C,caption={Ardunio code used for data collection},breaklines=True,captionpos=t,label=code:arduinomaster]{codes/3_load_cell.ino}

\subsection{arduino code to implement the hybrid active master-slave control}

%arduino code c
\lstinputlisting[language=C,caption={Ardunio code used for data collection},breaklines=True,captionpos=t,label=code:arduino]{codes/v1.ino}

\subsection{arduino code for calibrating each of the load cells}

%arduino code c
\lstinputlisting[language=C,caption={Ardunio code used calibrating each loadcell},breaklines=True,captionpos=t,label=code:calibration]{codes/calibration.ino}

\newpage


\subsection{python code for collecting data and storing it}
%pyton code listing not figure
\lstinputlisting[style=python,language=Python,caption={Python code used for data aquisition},breaklines=True,captionpos=b,label=code:py]{codes/force_torque.py}\newpage
\restoregeometry
\newgeometry{left=0.2cm,right=0.2cm,bottom=0.5cm} 
\section{Literature Review on Existing Bilateral Robots}\label{sec:bilatrob}
\href{https://docs.google.com/document/d/1uwM2z0r_MtfIO8P_6Rdd-_3ZBnt-zjLi/edit?usp=sharing&ouid=106338247166493930992&rtpof=true&sd=true}{link to full literature review}

\begingroup
\linespread{0.9}\sffamily\footnotesize
\begin{longtable}{|p{4cm}|p{3.8cm}|p{7cm}|p{4cm}|}
	\caption{Literature Review of Bilateral Rehabilitaion Robots}
	\label{tab:reviewbilateral}\\
	\hline	Overview&Aim&Methodology&Results\\
	\hline\hline
	\endfirsthead
	%&&&\\
	%\hline	
	\hline Overview&Aim&Methodology&Results\\
	\hline\hline
	\endhead
	An assessment of robot-assisted bimanual movements on upper limb motor coordination following stroke \cite{Lewis2009}. &The purpose of this study was to quantify the influence of
	robot-assisted bimanual movement on the timing of muscle activity
	and limb-robot interface forces during arm reaching tasks
	performed by individuals with hemiparesis.
	&15 subjects with poststroke hemiparesis, with all having suffered a \ac{cva} at least 12 months prior and had residual unilateral deficits in upper limb function. participated in 2 separate data collection sessions, they performed unimanual paretic limb, unimanual nonparetic limb and bimanual condition under each session. In the bimanual condition the task was performed in a mirror image symmetric manner.
	&For the first goal, when movement was specified by the nonparetic arm, force and emg profiles showed that the initial muscle activation was more synchronized with the limb motion during bimanual, robot-assisted movement.
	For the second goal, force and muscle activation profiles were similar in voluntary and robot-assisted bimanual movement. 
	Showing that bimanual robot-assisted movement training could provide additional therapeutic benefits beyond those of unimanual robot-assisted training in the long term.\\
	\hline
	Bilateral assessment of functional tasks for robot-assisted therapy applications \cite{Johnson2011b}.&To present a novel evaluation system along with methods to evaluate bilateral coordination of arm function on activities of daily living tasks before and after robot-assisted therapy&In the first study 10 healthy and 7 stroke subjects were included, with the stroke subjects having had a stroke occurring >6 months prior. While in the second study 4 stroke subject were included. 
	\ac{uefm} was used to describe motor control in the impaired arm, and the \ac{ueft} was used to describe functional disability levels.
	&It was found that the bias was able to measure accurately right and left arm kinematics during typical functional tasks and quantify movements during functional tasks pre- and post-robot therapy.
	Results from both studies showed the \ac{bias} system can accurately measure the kinematic wrist positions of all subjects regardless of impairments levels.\\
	\hline
	Rehabilitation robot with patient cooperative control for bimanual training of hemiparetic subjects \cite{Trlep2011}&To study the developments and validation of a bimanual training system that stimulates the use of both arms of the hemiparetic subjects.&The system is based on the HapticMaster, it is an admittance-controlled robot manipulator with a control loop rate $2500hz$ and consists of 3 active \ac{dof}, it was expanded with an extra joint to allow simulation of an active steering wheel which are bimanual handlebars mounted on the robot end-effector, and forces generated by each arm is measured by 2, 6- \ac{dof} force and torque sensors, a passive gravity compensation mechanism was implemented to compensate the weight of the subject’s upper extremities.&After the pilot study with 4 subjects with 8 trainings.
	The paper presents a system for unimanual and bimanual training and the system can be used as an evaluation device to monitor the patients progress and level of motor functionality.\\
	\hline
	Robot-assisted arm trainer for the passive and active practice of bilateral forearm and wrist movements in hemiparetic subjects \cite{Hesse2003}&To determine whether use of a robotic arm trainer for bilateral exercise in daily repetitive training for a 3-week period reduced spasticity and improved motor control in the arm of the severely affected, chronic hemiparetic subjects&The arm trainer was designed to allow the bimanual practice of a 1df pronation and supination movement as well as dorsiflexion and volarflexion of the wrist of the forearm, 3 operational modes were programmed which are passive mode with speed and \ac{rom}, active mode with mirror control and active mode as previous but the paretic arm has an isometric resistance to overcome.
	The trainer drivers can provide torques of up to 5Nm, a display shows the number of performed cycles.
	&The subjects had a positive impression on the therapy, and they reported that muscle tone reduction lasted a day. The end of the practice with the trainer resulted in a sustained reduction of muscle tone in most of the subjects and motor function improved in 5 of the subjects.\\
	\hline
	Incorporating haptic effects into three-dimensional virtual environments to train the hemiparetic upper extremity \cite{Adamovich2009}&Describes the design and feasibility testing of a robotic/virtual environment system designed to train the arm of persons who have had stroke.&The system makes use of a Haptic master a 3-dof admittance-controlled robot, a 3-d force sensor measures the external force exerted by the user also the velocity and position of the robot are measured at 1000hz and are used to generate reactive motion based on the properties of the virtual haptic environment. The \ac{haapi} allows the robot to programmed to produce haptic effects.&There was an improvement in the smoothness of the trajectories in all subjects.
	2 subjects improved their aggregate time to complete all 15 timed items.\\
	\hline
	Upper-limb powered exoskeleton design \cite{Perry2007}&To define the kinematics and dynamics of the upper limb during daily living activities to develop a anthropomorphic 7-\ac{dof} powered arm exoskeleton.&A preliminary study was carried out to understand the kinematic and dynamic requirements of an exoskeleton arm for functional use. Motions of the human arm was recorded during 18 \ac{adl} tasks which were divided into the following categories, general reaching tasks, functional tasks, eating and drinking, hygiene related tasks. &A robotic exoskeleton system was developed and tested using new techniques.\\
	\hline
	Computerized arm training improves the motor control of the severely affected arm after stroke a single randomized trial in two centers \cite{Hesse2005}&To compare a computerized arm trainer and electromyography-initiated electrical stimulation of the paretic wrist extensor in severely affected subacute stroke patients.&44 Subjects who had a stroke interval of 4 to 8 weeks were recruited to participate in the trial, the subjects practiced with an \ac{at} or \ac{es} with their paretic wrist extensors for 20minutes every day for 6 weeks. &There was improvement in the \ac{fms} for both groups, but significantly more in the AT group.\\
	\hline
	Repetitive bilateral arm training with rhythmic auditory cueing improves motor function in chronic hemiparetic stroke \cite{Whitall2000}&Investigation to determine if \ac{batrac} will improve motor function in the hemiparetic arm of stroke patients.&16 patients were recruited, they all experienced stroke >12 months previously. The training consisted of 20 minutes on the \ac{batrac} 3 times per week for 6 weeks.&14 patients completed the protocol and the \ac{uefm} showed significant improvements, only 3 subjects could extend their finger joints by >10degrees and wrist by \>20 degrees.\\
	\hline
	Driver’s SEAT: \ac{seat} \cite{Johnson1999}&Describes the design and philosophy of Driver’s \ac{seat}, a 1-dof robotic device that aims to promote coordinated bimanual movement.&The system consists of a motor, an adjustable-tilt, split steering wheel, a height adjustable frame, wheel position sensor, force sensors, \ac{sti} simulation hardware and experimenter computer.&It is a system designed to be used to stroke subjects with either right or left hemiplegia and was used in on 8 stroke patients.\\
	\hline
	Bilateral movement training with computer games for stroke rehabilitation \cite{King2010}&This paper describes two devices developed to use computer games during bilateral arm training.&Movement assisted gaming system is developed where the user sat with their arms strapped onto forearm supports and their hands rested on a joystick or palmar supports. &All the subjects reported they enjoyed using the device, and some described how their functional activity had improved.
	And from the \ac{fms} it was found that there was no deterioration because of the intervention.\\
	\hline
	Robotic stroke therapy assistant. \cite{Mahoney2003}&Describes the Design considerations and clinical outcome with regards to the phase 1 system of the \ac{arcmime}.&The \ac{arcmime} is made to be able to carry out the 4 control modes of the \ac{mime} system which are passive, active-assisted, active-constrained and bimanual mode. &There was no statistical difference between the \ac{arcmime}, and\ac{mime} forces directed toward the subjects. The subject feedback was neutral between both robots\\
	\hline
	A robotic system for upper-limb exercises to promote recovery of motor function following stroke. \cite{Lum1999}&To evaluate the therapeutic efficacy of robot-aided exercise for recovery of upper limb motor function following stroke.&Chronic stroke patients with \>6 months post \ac{cva} and were assigned randomly into either control or robot group, and they receive 24 one-hour sessions over 2 months.&The robot group subjects exhibited decreased resistance to some passive movements and improved performance of some active-constrained reaching movements post-treatment.
	Preliminary data from the ongoing clinical trial suggests robot-aided exercise has therapeutic benefits which can been seen improvement in active-constrained training tasks and the\ac{fma} of motor function.\\
	\hline
	Rhythmic bilateral movement training modulates corticomotor excitability and enhances upper limb motricity poststroke: a pilot study. \cite{Stinear2004}&To investigate whether repetitive bimanual coordinated movements enhanced upper limb corticomotor excitability and motor function poststroke.&9 poststroke patients ranging from 2 to 84 months post stroke were recruited; grip strength wrist flexor and wrist extensor forces were assessed before each mapping session.&5 patients increased their motricity scores over the period of the intervention.
	The experimental findings support the predicted increase in motricity and changes in Cortical maps excitability in response to \ac{apbt}. And there was no evidence that one of the 2 movement patterns were more effective than the other.\\
	\hline
	Priming the motor system enhances the effects of upper limb therapy in chronic stroke. \cite{Stinear2008}&To examine the effects of \ac{apbt}.&32 subjects, having had stroke 6 months prior were recruited and divided into either control or \ac{apbt} group. All subjects were given a set of wooden blocks to use for their intervention for which they had to pick up and transport the block 20cm. the \ac{apbt} group made use of a \ac{apbt} device for 15 mins.&All subjects were stable at baseline and improved immediately after intervention, but only those primed with \ac{apbt} before motor practice showed sustained improvement in upper limb motor function.\\
	\hline
	The \ac{mime} robotic system for upper-limb neuro-rehabilitation results from a clinical trial in subacute stroke. \cite{Lum2005}&Presents the results from a randomized, controlled clinical trial of the \ac{mime} robotic device for shoulder and elbow neurorehabilitation in subacute stroke patients.&The paretic arm was strapped to reduce wrist and hand movement, the Puma 560 robot is attached to the splint, there are 4 modes of robot-assisted movement.
	Subjects had a \ac{cva} 1-5 months prior, intervention consisted of 1 hour treatment 15 times, the robotic treatment used was similar to a previous study and there were 3 groups for this study, robot-unilateral, robot-bilateral, robot-combined and control group. 
	Robot-unilateral group performed 
	Subjects were tested before and after each assignment the FM assessment
	&Both robot groups had significant gains in the proximal and distal \ac{fms}, \ac{mss} movement, motor power and \ac{fim}. 
	The robot-combined group had a better Ashworth score compared to the robot-unilateral group.
	Compared to the control group the robot-combined group had greater gains in the proximal \ac{fms} and \ac{mss}, and the control group did better than the robot-unilateral group.
	The robot-bilateral group did poorer on motor power and \ac{fim} compared to the robot-unilateral and control groups.\\
	\hline
	Development of robots for rehabilitation therapy: the palo alto VA/Stanford experience. \cite{Burgar2000}&To summarize the development and clinical testing of 3 mechatronic systems for post-stroke therapy.&The 3 mechatronic system discussed are upper limb patient-controlled manipulation orthosis, \acf{mime} and a third gen version of the \ac{mime} system.&\\
	\hline
	Robot-assisted upper-limb therapy in acute rehabilitation setting following stroke: Department of Veterans Affairs multisite clinical trial. \cite{Burgar2011}&To evaluate whether the \ac{mime} can facilitate similar or greater motor recovery as the same of early hands-on therapy &54 Subjects entered the study between 7-21 days after stroke. Divided into 3 groups, robot low does, robot high does and control. Control and robot low receiving 15 1 hour sessions over 3 weeks and robot high receiving 30 1 hour sessions over 3 weeks. &At post treatment the robot-hi and control groups had higher \ac{fms} than the robot-lo group with both groups showing similar results. But at the 6 months follow up the robot-lo and control group results were similar but the robot-hi results much higher than both, and the robot-lo group showing large improvements compared to both groups.\\
	\hline
	Bilateral upper limb trainer with virtual reality for post-stroke rehabilitation: case series report. \cite{Sampson2012a}&To investigate the body function effects and motivational effects of a combined therapy based on several stroke rehabilitation concepts.&The system comprises of the \ac{built} with \ac{vr}, the \ac{built} facilitates bilateral movement in the horizontal plane of the affect arm, forearm supports are used to hold the upper limb in places and allow the patient to grip with a joystick style grip on the handle. The BUiLT provides shoulder and elbow flexion/ extension, shoulder adduction/ abduction, external/internal rotation, and combined movement patterns.&The results showed increase in \ac{uefm} scores across all subjects, there was also increase in joint movements in most of the subjects.
	Overall, the results from the intervention shows there was a trend of positive effect on UL function from the \ac{built} + \ac{vr}. Isometric strength also showed a mostly increasing result across all the subjects.
	The question form results indicate that the system highly motivated the subjects to exercise, and they were able to complete the full regiment.
	\\
	\hline
	Robot-assisted movement training compared with conventional therapy techniques for rehabilitation of upper-limb motor function after stroke. \cite{Lum2002}&To compare the effects of robot-assisted movement training with conventional techniques for rehabilitation of the upper-limb motor function after stroke.&This paper makes use of the \ac{mime} system. 30 Subjects were enrolled in study if they had a diagnosis of a single \ac{cva} and were more than 6 months post \ac{cva}. Subjects were assigned to control or robot group, each group received 24, 1-hour treatment sessions over 2 months.
	The robot group emphasis was on targeted reaching movements, while for the control was on re-education of muscles using a sensorimotor approach to control motor output.
	&Compared with conventional treatment of equal intensity and duration, the robot-assisted movements had advantages after 2 months of treatment in terms of decreasing impairment, improving strength and increasing reach extent.\\
	\hline
	Machines to support motor rehabilitation after stroke: 10 years of experience in Berlin. \cite{Hesse2006}&Presents the devices and related clinical studies for the motor rehabilitation of these upper limbs.&\textbf{Bi-Manu-Track}: it is a 2x1 \ac{dof} than enables hemiparetic patients to bilaterally practice 2 different movement cycles, forearm pronation/ supination and wrist flexion and extension. &The robot-trained group had superior results at the end of the study and at a 3-month follow up\\
	\hline
	Unilateral and Bilateral upper-limb training interventions after stroke have similar effects on bimanual coupling strength. \cite{VanDelden2015}&To determine whether the degree of coupling between both hands is higher after bilateral than after unilateral training. &60 subjects were admitted into the trial with an average time after stroke of 9.3 weeks and divided into 3 groups \ac{mcimt} which is the unilateral group and involved unilateral repetitive tasks, modified \ac{mbatrac} and dose matched control treatment. All subjects received 1-hour therapy sessions 3 days per week for 6 weeks.
	The study makes use of 4 tasks which are bimanual coordination task, kinesthetic tracking tasks, unimanual motor task and unimanual reference task.
	&At the follow-up, the \ac{dmct} group showed significantly stronger intended coupling in the in-phase kinesthetic tracking test than at post intervention.
	There was observed improvements in the paretic hand movements after \ac{mbatrac}.
	\\
	\hline
	Comparing unilateral and bilateral upper limb training: the \ac{ultra}-stroke program design. \cite{VanDelden2009}&Describes the design of a single-blinded randomized clinical trial to access the effectiveness 2 new intervention techniques in subacute stroke patients and to examine the changes in how sensorimotor functioning relate to changes in stroke recovery mechanisms.&This paper makes use of both \ac{batrac} and \ac{cimt}. 
	60 subjects were recruited with the criteria of having a CVA within 6 months. The trail is to last for 6 months.
	The \ac{mbatrac} group receives 1-hour sessions, 3 days a week for 6 weeks. A computer is connected to the potentiometers on the devices measures movements and provides feedback, this computer is also used to start each exercise. &No significant difference between groups on the primary and secondary outcome measures between groups at posttest ad follow-up. All groups demonstrated significant improvement on the action research arm test after intervention during the 6 weeks follow up\\
	\hline
	Quantifying learned non-use after stroke using unilateral and bilateral steering tasks. \cite{Johnson2011}&To examine whether the behaviour of underutilizing the impaired arm slowing down re-acquisition of bilateral coordination can be studied and quantified using the TheraDrive.&The TheraDrive is used as the experimental apparatus for the study, the system consists of a Logitech force-reflecting wheel mounted on a height adjustable metal frame, the drive is connected to a UniTherapy software platform that records the angular movement of the wheel.&The results showed higher average errors for the impaired arm tracking than non-dominant arm tracking, confirming it is not as efficient in performing tracking tasks when compared to the dominant arm in stroke survivors.
	And it was also found that when the paretic arm is highly involved the performance of bilateral tracking tasks will be affected and bilateral tracking errors will be like unilateral tracking errors with the impaired arm.
	\\
	\hline
	Experimental results using force-feedback cueing in robot-assisted stroke therapy. \cite{Johnson2005}&To evaluate a novel force-feedback and reinforcement strategy aimed at limiting the tendency of stroke survivors with hemiparesis to overuse their less-affected arm.&8 subjects participated with a mean time after stroke of 4.8 years. The Drivers \ac{seat} system was used for this study. \ac{semg} activity for the 7 upper limb muscles were recorded. 
	There were 4 conditions data was obtained from the wheel which are: N-bi, A-bi, N-WA, A-WA. 4 trials were completed for each steering task. Subjects were tasked to track a target on the screen while steering through 15 RT, ST and LT segments on the track.
	All subjects had data collected for the 4 modes and the data was used to evaluate the effect of the corrective force
	& Corrective force cues had some effect on the weak arm participation rate of all subjects during bilateral steering of the preview track.
	It was found that a significant force cue effect on the impaired arm of subjects during particular movements in one therapy session with the device, indicating that the subjects used their impaired arm differentially in each of the force-cue condition
	\\
	\hline
	Bilateral arm training with rhythmic auditory cueing in chronic stroke: not always efficacious. \cite{Whitall2000}&To determine whether the reported results from a modified form of \ac{batrac} could be replicated.&15 Subjects participated in the trial had a stroke at least 6 months prior. The \ac{uefm} was used for assessment.
	The intervention involved the subjects moving the \ac{batrac} for 5 minutes with 10 minutes break. And this was done for 2 weeks with 2.25-hour sessions.
	&It was found that an increase in training speed did not have a consistent effect on functional outcomes but contributed to perceived ability and use of ability. 
	A larger proportion of subjects with left hemisphere damage responded on the \ac{uefm} and \ac{wmft} scales.
	It was found for subjects with lower initial \ac{uefm} their gains were much larger than the other group
	\\
	\hline
	25 post stroke shoulder-elbow physiotherapy with industrial robots. \cite{Toth2006}&Describes the REHAROB therapeutic system.&2 ABB industrial robots were selected for delivery of exercises, with one connected to the upper arm and the second to the lower arm. A Orthosis is developed to connect the patient’ upper and lower arm to robot, it is of an exoskeleton form.&With the \ac{mas} scoring there was improvement in both shoulder adductors and elbows flexors in all but 1 subject. The range of motion of all subjects increased at the end of the trial same with the \ac{fim} and Barthel index.
	it was also found that the patients were not afraid of the robot and the therapists were able to learn how to use the system, and the system was working reliably.
	\\
	\hline
	Effects of robot-assisted upper limb rehabilitation on daily function and real-world arm activity in patients with chronic stroke: a randomized controlled trial. \cite{Liao2012}&To compare the outcome of robot-assisted therapy with dose matched active control therapy using accelerometers to study functional recovery in chronic stroke patients&20 subjects were recruited for the study, all having a \ac{cva} after 6 months. Subjects were randomly assigned to robot-assisted therapy or to dose matched active control therapy.&There were significant benefits of the robot-assisted therapy compared to the active control group on the amount and quality of functional arm activity for the hemiplegic hand in the living environment. 
	This study demonstrated that an accelerometer could be a suitable measure of treatment efficacy in robot-assisted therapy. Also showed that robot-assisted therapy combined with 15 minutes of functional activity has great benefits on real world arm activity
	\\
	\hline
	Kinematic data analysis for post-stroke patients following bilateral versus unilateral rehabilitation with an upper limb wearable robotic system. \cite{Kim2013}&This paper reports on a randomized clinical trial utilizing a complete robot-assisted rehabilitation system for the recovery of the upper limb function in patients’ post-stroke.&The \ac{ulexo} was used for this study.
	25 subjects joined the trial having passed the requirements of having a \ac{fms} between 16 and 39 and more than 6 months post stoke. They were randomly divided into 3 groups: unilateral robot training, bilateral robot training and usual care.
	&It was found that bilateral movement training delivered a better rehabilitation result to subjects except in the reaching exercise.
	The \ac{fms} showed that there is no significant difference between the 2 training methods.
	It was found that when considering individual evaluation metric that the bilateral training scheme delivered better rehabilitation result with respect to the wrist joint movement.
	\\
	\hline
	MIT-MANUS: A Workstation for manual therapy and training I. \cite{Hogan1992}&This paper presents some recent work on the development of a workstation for teaching and therapy in manual and manipulative skills.&The MANUS has 5 \ac{dof}, a direct-drive five bar linkage SCARA mechanism provides 2 translational \ac{dof} for the elbow and forearm motion. A differential mechanism driven by geared actuators provides 3 \ac{dof} of wrist motion: extension-flexion, abduction-adduction and pronation-supination.&The proposed system was built and tested in multiple future clinical trials\\
	\hline
	Combined transcranial direct current stimulation and robot-assisted arm training in subacute stroke patients: an exploratory, randomized multicentre trial. \cite{Hesse2011}&To test the combination of upper arm interventions were tested in a double-blind randomized trial.&96 patients were randomized for 6 weeks and split into 3 groups which received anodal stimulation, cathodal stimulation, and sham stimulation, respectively. The inclusion parameters were that the patients were at least wheelchair mobile and had a severe flaccid \ac{ul} paresis with minimal hand and finger extensor activity. &All patients improved their \ac{fms} significantly over time, and between-group differences did not occur at any time. 
	Contrary to the hypothesis, nether stimulation affected the bilateral robot-assisted arm training.
	\\
	\hline
	Guidance-based quantification of arm impairment following brain injury: a pilot study \cite{Reinkensmeyer1999}&Reports the design and preliminary testing of a device for evaluating arm impairment after brain injury.&4 control subjects with no known history of neurological deficits and 4 brain-injured subjects participated in the trial. The device to be used was the \ac{arm} guide that uses a custom splint that rides along a linear constraint. &The results support the hypothesis that abnormal synergies would manifest themselves as distinct patterns of constraint force during guided movement.\\
	\hline
	Upper and lower extremity robotic devices to promote motor recovery after stroke- recent developments. \cite{Schmidt2004}&This paper presents clinically viable devices for upper and lower extremity rehabilitation.&The upper rehabilitation device discussed is the Bi-Manu-Track robot, which enables the bilateral passive and active practice of 2 movements which are: forearm pro-supination and wrist flexion and extension in a mirror/ parallel way. &The study shows the viability of robotic devices in rehabilitation of both upper and lower limbs, with subjects involved the trails showing improvements.\\
	\hline
	Design and development of a hand robotic rehabilitation device for post stroke patients. \cite{Rashedi2009}&Describes the design of a robotic device for rehabilitation of upper limbs of post stroke patients.&The apparatus was designed to provide the passive or active unilateral or bilateral therapeutic exercises for the upper limb in 2 operation states. Pronation/supination of forearm and flexion/extension of wrist. A special form of handles allowed them to be used for both movements without the need to be exchanged.&The device was tested on a female patient with left sided paretic arm, who utilized all 3 modes, the device was able to produce an accurate mirror image position tracking of the master limb.\\
	\hline
	A new haptic workstation for neuromotor rehabilitation. \cite{Casadio2006}&It describes a new robotic workstation for neurological rehabilitation.&Solutions based on linear motor tables with non-reversible hears and typical robot designs based on kinematic chains were avoided to meet the requirement of back-drivability. &A prototype of the system was constructed, and the design assumptions were verified on it, and the authors are confident that it’s operational range is consistent with the typical human motions of the upper limb and it fit for robot therapy.\\
	\hline
	%&&&\\
	%\hline
\end{longtable}

\endgroup
\restoregeometry
\newgeometry{left=0.2cm,right=0.2cm,bottom=0.5cm}
\section{Literature Review on Clinical Trials of Bilateral Robots}\label{sec:clinical}


\begingroup
\linespread{0.9}\sffamily\footnotesize

\begin{longtable}{|p{3cm}||p{2cm}|c|p{3cm}|c|}
	\caption{Literature Review on Clinical Studies and their results}
	\label{tab:reviewclincal}\\
	\hline 	Paper and number & Number of patients & Stage of recovery & Range of weeks after stroke for recruitment & Use of FES \\
	\hline\hline
	\endfirsthead
	\hline 	Paper and number & Number of patients & Stage of recovery & Range of weeks after stroke for recruitment & Use of FES \\
	\hline\hline
	\endhead
	
	
	\cite{Lewis2009} & 15 & Chronic & $>$12 months & no \bigstrut\\
	\hline
	\cite{Johnson2011} & 11 & Chronic & $>$6 months & yes \bigstrut\\
	\hline
	\cite{Trlep2011} & 4  & Chronic & 5-13 years & no \bigstrut\\
	\hline
	\cite{Hesse2003} & 12 & Chronic & 6-16 months & no \bigstrut\\
	\hline
	\cite{Adamovich2009} & 4  & Chronic & 1-8 years & no \bigstrut\\
	\hline
	\cite{Hesse2005} & 44 & Early sub-acute & 4-8 weeks & no \bigstrut\\
	\hline
	\cite{Whitall2000} & 16 & Chronic & $>$6 months & no \bigstrut\\
	\hline
	\cite{King2010} & 16 & Chronic & $>$6 months & no \bigstrut\\
	\hline
	\cite{Mahoney2003}  &13,21,4 & Chronic & $>$6 months & no \bigstrut\\
	\hline
	\cite{Liu2010} & na & Chronic &>6 months & no \bigstrut\\
	\hline
	\cite{Stinear2004} & 9  & 5 Chronic and 4 early sub-acute & 2-84 months & no \bigstrut\\
	\hline
	\cite{Stinear2008} & 32 & Chronic & 6-144 months & no \bigstrut\\
	\hline
	\cite{Lum2005} & 23 & Early sub-acute to chronic & Average 6.2-13 weeks & No \bigstrut\\
	\hline
	\cite{Burgar2000} & 13,21 & Early subacute to chronic & 1-45 months, $>$6 months &  \bigstrut\\
	\hline
	\cite{Burgar2011}1 & 54 & Early sub-acute & 7-21 days  & no \bigstrut\\
	\hline
	\cite{Sampson2012} & 5  & 1 early sub-acute and 4 chronic & 9-64 weeks & No  \bigstrut\\
	\hline
	\cite{Lum2002} & 30 & Chronic & $>$6 months & No \bigstrut\\
	\hline
	\cite{Hesse2006} & 44 & Early subacute & 4-8 weeks & no \bigstrut\\
	\hline
	\cite{VanDelden2012} & 60 & Early sub-acute & Average 7.8-11.1 weeks & no \bigstrut\\
	\hline
	\cite{Johnson2011b} & 7  & Chronic & $>$6 months & no \bigstrut\\
	\hline
	\cite{VanDelden2009} & 60 & Early sub-acute & $>$6 months & No \bigstrut\\
	\hline
	\cite{Johnson2005} & 8  & Chronic & 1-10 years & No \bigstrut\\
	\hline
	\cite{Richards2008} & 15 & Chronic & $>$6 months & No \bigstrut\\
	\hline
	\cite{Toth2006} & 8  & Early and late subacute to chronic & 8 weeks to  9 years & no \bigstrut\\
	\hline
	\cite{Liao2012} & 20 & Chronic & $>$6 months & No \bigstrut\\
	\hline
	\cite{Kim2013} & 25 & Chronic & $>$6 months & No \bigstrut\\
	\hline
	\cite{Hesse2011} & 96 & Early sub-acute & 3.4-3.8 weeks & No \bigstrut\\
	\hline
	\cite{Schmidt2004}& 12,55 & Chronic and acute & >6months and 0-6 months & no \bigstrut\\
	\hline
	\cite{Masiero2007}& 35 & Acute & $>$1week & No \bigstrut\\
	\hline
	\cite{Buschfort2010} & 119 & Sub-acute and chronic & Within 8 months & No \bigstrut\\
	\hline
	\cite{Whitall2011} & 111 & Chronic & $>$6 months & No \bigstrut\\
	\hline
	\cite{Hijmans2011} & 13 & Chronic & $>$6 months & No \bigstrut\\
	\hline
	\cite{Stoykov2010} 451 & 19 & Sub-acute & 1-5 weeks & No \bigstrut\\
	\hline
	\cite{Chang2007}  & 20 & Acute & $<$6 months & No \bigstrut\\
	\hline
	\cite{Diez2018}& 6  & chronic & na & no \bigstrut\\
	\hline
	\cite{Hsu2019} & 43 & Chronic & 14.2 moths & no \bigstrut\\
	\hline
	\cite{Simkins2016} & 15 & Chronic & $>$6 months & no \bigstrut\\
	\hline
	\cite{Reinkensmeyer1999}& 4  & Chronic & 2-16 years & No \bigstrut\\
	\hline
	\cite{Hesse2008} & 54 & Early sub-acute & 4-8 weeks & Yes for control \bigstrut\\
	\hline
	\cite{Wu2012} & 42 & chronic & $>$6 months & No \bigstrut\\
	\hline
	\cite{Wu2013} & 53 & Chronic  & 6 months to 5 years & no \bigstrut\\
	\hline
	\cite{Barsotti2016} & 2  & chronic & na & no \bigstrut\\
	\hline
	\cite{Hsieh2017}& 31 & subacute & $>$ 6 months, average 2.4 months & no \bigstrut\\
	\hline
	\cite{Hung2019} & 44 & Chronic & $>$6 months & no \bigstrut\\
	\hline
	\cite{Hung2019b} & 30 & chronic & $>$6 months & no \bigstrut\\
	\hline
	\cite{Chen2016} & 7  & Chronic & $>$6 months & No \bigstrut\\
	\hline
	\cite{Squeri2009} & 4  & Chronic & $>$6 months & No \bigstrut\\
	\hline
	\cite{Hsieh2017} & 34 & Chronic  & $>$6 months & No \bigstrut\\
	\hline
	\cite{Straudi2016} & 23 & 9 subacute, 14 chronic & Average per group, 40.7 and 78.2 weeks & No. \bigstrut\\
	\hline
\end{longtable}

\endgroup
\restoregeometry
